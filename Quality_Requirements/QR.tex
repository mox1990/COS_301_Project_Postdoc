These are the quality requirements as specified by the client and are also in the order of precedence. Where the first takes the highest precedence.

\subsubsection{Usability requirements}

\begin{flushleft}

This is first and foremost quality requirement stipulated by the client. The primary language of the system will be South African English. Any other language support is not considered part of the requirements but the system will be designed to allow for such development in the future.\\

\vspace{0.1in}

The system's UI will only consider 2 types of user categories with regard to usability:

\begin{itemize}

\item\textbf{Trained user:}

This type of user will have to have training in understanding how the system functions and how to use it. Their computer skills will be assumed to be in the range of basic to intermediate. Thus the user interface can allow for certain complexities but these complexities must be kept at a minimal. These user will be regarded as system administrators or specialised users of the system. The stakeholders who fall under this category is the DRIS staff members overseeing the application process and any technical or maintenance support users.

\item\textbf{Normal user:}

This type of user will have no training with regard to the system. Their computer skills will be assumed to be none or minimal. Therefore the UI that they will have access to will be simplistic and will be as user friendly as possible. The stakeholders that fall under this category will be Prospective fellows, Research fellows, Referees, Grant holders, HODs, Dean's office members and Post-doctoral committee members.

\end{itemize}

In order to quantify the above the SoftServe group will need to provide the system with a uniform user experience that is as simplistic and as visually oriented as possible without negating the below requirements severely. In addition to this any form of help text or instruction should be written in as natural, uncomplicated and unambiguous manner as possible. A further quantification of this requirement can be seen as finding the least number of clicks a user has to perform in order to perform any operation. Thus to meet this requirement it should be considered local tasks should take a maximum of five clicks and distant tasks should take up to a maximum of 8 clicks. Where local refers to tasks that are related to the current section begin accessed with in the system. These clicks should not include the clicks need to enter various lengthy form data.

\end{flushleft}

\subsubsection{Security requirements}

\begin{flushleft}

The system will need to be fully secured since it deals with confidential information such as person information, application statuses, financial data and meeting information. Also since the systems main goal is to provide stable and audible application and renewal process flow the system may not be vulnerable to data tampering or any tampering whatsoever.\\
\vspace{0.1in}

The system will have to provide different security roles to the registered users on the system. Any number of roles should be assignable to any user by a system administrator to allow for flexibility.\\
There will also need to be predefined roles so to allow a stakeholder from list of stakeholders in the vision and scope document, to only have the ability and access to perform the actions as defined by the vision and scope document.\\
The system administrator should be able to access all the sections in the system and should be able to modify them except where only the system itself has that right.\\
Another security requirement is that the audit trail must not be allowed to be modified by any human user and only allow the system to be able to insert data into the audit log. Further the audit table should only be query-able by a user with the correct security role.\\
To further quantify this requirement the system must provide some means of a centralised user authentication that acts as a gateway for all users of the system. Also every time some action is made by the user the user needs to be re-authenticated to insure that the user does have the correct privileges and if any user account changes where made it takes immediate affect. \\
Any form of account retrieval token or on-demand user account creation tokens must expire within a specified time period. For account retrieval token this should be 1 hour. For on-demand user account creation this should be about 2 days.
\vspace{0.1in}
This requirement will influence the usability requirement negativity as it is known that the more secure a system becomes the less user friendly it becomes because there will be need for more user forms and interactions. But the impact of this requirement must be minimized as much as possible. Therefore the SoftServe group will try to balance the two requirements in such a way that neither of the two is negated. Further this requirement will also negatively impact performance as it will result on more code command execution per action that needs to be secured.
\end{flushleft}

\vspace{0.1in}

\subsubsection{Audit-ability:}

\begin{flushleft}

This is seen as the third most important requirement according to the client. The system needs to provide some form of an audit trail of all critical actions that occur in the system. Critical actions are considered: user account management operations, login action, logout action as well as any operation by a user that leads to a change in the data stored by the system. The Audit trail will be in the form of a read-only table stored in the database as stipulated in the security requirements above.\\

To quantify this requirement, the system will need to have an active audit log that can capture all the actions, that meet the list of critical actions, of each user that uses the system. With regards to usability, the feature needs to run in the background and not be noticeable by any user while they are using the system except if it is deliberately queried.

\vspace{0.1in}
This requirement will have a negative impact on performance as it will be adding more overhead the executions of audit-able actions. But since not all the actions are audit-able it should not have a to significant impact. Though the audit logging action should be made as light weight as possible. Also adequate storage will be need since such entries can easily grow exponentially which can lead availability and other performance issues.  

\vspace{0.1in}

\end{flushleft}

\subsubsection{Scalability requirements}

\begin{flushleft}

As research is still viewed as small field that is slowly growing by the client this is only the fourth most important requirement for the system. The current aim is to create a scalable system that can support 50 to 500 applicants per year with possible growth to a 1000 or more. This is in line with the figures given by the client and the growth, over the next ten to twenty years, in the research sector of the University of Pretoria.\\
\vspace{0.05in}

The system needs to be scalable in regard to the following factors:
\begin{itemize}

\item\textbf{Performance:} This aspect of the system regards the speed and responsiveness of the system.
The system needs to handle large report queries in less than 10 seconds. It should be able to handle any application approval service processing in less than 3 seconds. It should be able to send emails instantly from the system side. Note this may be difficult to measure due internet bandwidth and receiving mail server constraints. Login, logout and user authentication should be handled in less then a second by the system this does not  considering the transfer of data to and from the server which is depended on the network connection speed.\\

\item\textbf{Storage:} This aspect of the system regards the growth and shrinking of the data that is stored by the system.
The system will need to be able to handle a database that is in the range of 1 GB to 15 GB that has the potential to grow even larger. The reason for this stems from the fact that the system will need to store a large amount of active data over a period of about 3 to 8 years before it can be archived.
\\Another requirement of the system is that it will need to support archival functionality and archived data that will eventual store all the old data from the active database for as long as time period as possible. Since the archival database should only grow it will need to be about 15 times larger than the active database. Due these reasons the calculation of how much space should be required will only be possible once the system is implemented and practically tested.\\The suggested linear formula to the above analysis is as follows: 
\begin{equation}
A \times ( (Xna \times Nna) + (Xra \times Nra) ) = B
\end{equation}

A = Number of years for which archival or active support is intended\\
B = Number of bytes needed on average for A years\\
Xna = Average number of bytes per new application\\
Xra = Average number of bytes per renewal application\\
Nna = Number of new applications a year\\
Nra = Number of renewal applications a year\\

\item\textbf{Concurrency:} This regarded as the amount of active users on the system at the same time.
The system will need to support at least 100+/- concurrent users efficiently and effectively since the system requires multiple stakeholders to part take in the application process while there can be multiple applications occurring at the same time.\\

\end{itemize}

This requirement will require that the code undergo optimization and as well as certain functionality which can reduce the maintain-ability of the code itself also it can lead to reduction in terms of system robustness.

\end{flushleft}
\vspace{0.1in}

\subsubsection{Availability:}

\begin{flushleft}

The system's availability on designated networks will depend on the availability of the servers of the hosting service's that will be used to host the system.\\
In order to quantify this it should be seen as follows: If the hosting service's servers hosting the system are active and provide access over a designated network then the system must be available over that designated network. Where the designated networks are defined as the internet and/or the local network where in the server of the hosting service is located.

\vspace{0.1in}
%Say something about the trade off of avaibility
\end{flushleft}
\newpage
\subsubsection{Robustness:}

\begin{flushleft}

This requirement was identified by the SoftServe group as being very important since the system will be receiving and handling large amounts of sensitive data. Thus the integrity of the data must be insured. To do this the system needs to be robust in in terms of its data validation routines, data storage routines, security routines, and system integrity checks. 

These four aspects of robustness are quantified as follow:
\begin{itemize}
\item \textbf{Data validation}: This aspects refer to the high level data validation that occurs at a user interface level. All data that is manually imputed by a user, such as text, numbers, emails, passwords, etc. Must be checked to see if they meet constraints imposed on them through the use of data type checking and regular expression analysis. Any section in the system that requires certain data to completed by the user before performing a action must be checked for completeness before allowing the user to perform the action. Where completeness means the data is not empty and validate according to constraints imposed on it. Data is only sent to lower levels of the system after the data has been validated.
\item \textbf{Data storage}: When the data is actually stored the data needs to be validated against a set of low-level storage rules and constraints. Many DBMS packages allow this type of validation through data typing and allowing the database administrator to a specify constraints. Note that this is validation is independent from the validation that occurs at the user interface level.
\item \textbf{Security}: As mentioned above in the security requirements, the re-authentication of users is seen as a robustness check. Another robustness check for security is that a particular user may only have one of either a account retrieval token or on-demand user account creation token active at a time else it is regarded as fault.
\item \textbf{System Integrity}: If any malfunctioning modules are detected within the system any user action will not be completed and the user will be notified immediately with the appropriate error message. So prevent any data corruption due to system faults.
\end{itemize}

\vspace{0.1in}

\subsubsection{Testability:}

\begin{flushleft}

The system must be testable in order to show that the system meets the clients expectations. This will be done using user acceptance testing, integration testing and unit testing. The test plan will be laid out in the Non-Functional and Functional testing documents of this project. Each of which will provide a set of tests to be performed in order to verify the systems compliance. Therefore each non-functional and functional requirement needs to be quantified in a manner that allows the SoftServe group to test for it in the system. This requirement will help ensure the other requirements are met and all in all improve the system as a whole. Below are the suggested methods quantifying the tests for each type of test:\\


\vspace{0.1in}

Unit testing will test each module and its unit in regard to:
\begin{itemize}

\item\textbf{Pre-conditions} of each unit contained in a module.
\item\textbf{Post-conditions} of each unit contained in a module.
\item\textbf{Result and request objects} of each unit contained in a module.

\end{itemize}

The integration tests will test the system with regard to:

\begin{itemize}

\item \textbf{Module communication} between units and if their interfaces match the expected interfaces. 
\item \textbf{Functionality} of the system in contrast to the required functionality.
%\item\textbf{Offline:} This is the initial phase of testing and debugging which will be done with pseudo data.
%\item\textbf{Online:} This is the final phase of testing and debugging which will be done with active real time data.

\end{itemize}

User acceptance testing will test the whole system in regard to:
\begin{itemize}

\item\textbf{Non-functional requirements}: If the system meets the non functional requirements as stipulated by the architectural requirements documents.
\item\textbf{Functional requirements}: If each of the expected system functional requirements are met as stipulated in the functional requirement and application design document.
\item\textbf{Client acceptance}: If the system is at a level that client accepts the system and wishes to deploy it.

\end{itemize}

\end{flushleft}



\end{flushleft}


\subsubsection{Flexibility:}

\begin{flushleft}

The system should be designed in such a way that adding new features can be done without extensive code restructuring or reverse engineering. The process of adding new features should require the programmer working on the system to easily create a new module for the new functionality. For this to be realised the system will should use appropriate design patterns that allow this. For example the strategy, dependency injection and command design patterns. 

\vspace{0.1in}
This requirement will lead to an increase in complexity as implementing some of the design patterns will still require satisfying the large set of requirments (including usability).
\end{flushleft}


\subsubsection{Re-usability:}

\begin{flushleft}

The system's components should be designed in such away that they can be reused in different sections in the system or even in external systems. For example the user access control module should be portable to other systems that can be developed.

\end{flushleft}

\vspace{0.1in}

\vspace{0.2in}
\newpage
\subsubsection{Maintainability:}

\begin{flushleft}
The system should be maintainable in regards to 2 aspects: 
\begin{itemize}
\item \textbf{System housekeeping}: This aspect revolves around the actually administration of the system once it is deployed. The system should prevent the system administrator from unnecessary worrying about automated tasks. But the system should allow the system administrator to be able to manage user data, perform backups manually, archival routines and system configuration manually. 
\item \textbf{System development}: This aspect revolves around the developers of the software. The system must be developed in such a way that the source code is easy to maintain and highly readable such that debugging and patching difficulty can be reduced. Also this ensures that the code can easily given to a third party if need be. The coding documentation standard that must be adhered is that stipulated by the JavaDoc standard. The coding style convention must be defined in such away that the code is easy to read and structured in a logical manner. The identifiers used must be named appropriately and must follow the camelCase standard except where the identifiers are constants in which case the entire name will be in capital letters. The style suggested is a variant of the Allman style:
\begin{lstlisting}
public void run()
{
	/*This is a multi line commment*/
	//This is a single line comment
	
	for (int i = 0; i < x; i++)
	{
		int x = 0;
		int y = 0;
		while (x == y)
		{
			if (x == y)
			{
			    something();			    
			}
			else
			{
				somethingElse();
			}
		}
	}
}
\end{lstlisting}
\end{itemize}


\end{flushleft}

\vspace{0.1in}