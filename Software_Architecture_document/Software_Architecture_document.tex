\documentclass[12pt]{article}

\usepackage[english]{babel}
\usepackage[utf8x]{inputenc}
\usepackage{pdfpages}
\usepackage{lastpage} % Required to determine the last page for the footer
\usepackage{extramarks} % Required for headers and footers
\usepackage{graphicx} % Required to insert images
\usepackage{listings} % Required for insertion of code
\usepackage{courier} % Required for the courier font
\usepackage{color}
\usepackage{grffile}
\usepackage{float}

\usepackage[a4paper, total={6in, 8in}]{geometry}

% Margins
\topmargin=-0.45in
\evensidemargin=0in
\oddsidemargin=0in
\textwidth=6.5in
\textheight=9.0in
\headsep=0.25in
\fboxsep=0mm%padding thickness
\fboxrule=2pt%border thickness

\linespread{1.1} % Line spacing

\newcommand{\Title}{Software architecture document} % Assignment title
\newcommand{\Class}{COS\ 301 Final year project} % Course/class
\newcommand{\pd}{Post-Doctoral}
\newcommand{\ssr}{Soft\color{green}{Serve }\color{black}}
\newcommand{\version}{2.0}
\newcommand{\iteration}{5}
\newcommand{\client}{Ms. Cathy Sandis (UP DRIS)}
\newcommand{\project}{Post-Doctoral Application Management System}
\newcommand{\repo}{https://github.com/mox1990/Project-Postdoc.git}
\begin{document}

\vspace{4em}

\begin{center}%

\begin{figure}[ht!]
\centering
\includegraphics{../Images_Docs/logo.png}
\end{figure}
\LARGE \bf \Class \\[0.25em]
\LARGE \bf \project \\[1em]
\LARGE \bf \Title \\[0.25em]
\large \bf \today\\
\bf Version \version\\
\bf Iteration \iteration\\[0.5em]
\Large \bf Prepared for \client\\
\Large \bf by
\Large {\bf \ssr Group }\\[0.5em]
\LARGE {\bf Group members}\\[0.25em]
\large
Kgothatso Phatedi Alfred Ngako (12236731) \\[0.5em]
Tokologo “Carlo” Machaba (12078027) \\[0.5em]
Mathys Ellis (12019837) \\[8em]

\end{center}%

%\newpage
%{\LARGE \bf Change log}\\[2em]

\begin{center}
\begin{tabular}{|l|p{1.4cm}|p{8cm}|p{2.8cm}|}
\hline
\multicolumn{4}{|c|}{\bf Change log} \\
\hline
 Date & Version & Description &  Person \\
\hline
03/10/2014 & v 0.0 & Combined architectural specification and architectural requirements specification & Mathys Ellis \\
\hline
03/10/2014 & v 1.0 & Complete restructuring and most of the editing and additions  & Mathys Ellis \\
\hline

%\end{tabbing}
\end{tabular}
\end{center}
\newpage
\tableofcontents

\listoffigures
\newpage
\section{Project Repository}
\textbf{\repo}
\newpage
\section{Document description:}


\subsection{Document purpose:}
\vspace{0.2in}
The Software architecture document provides the architectural requirements that the project will need to follow as well as a set of quality requirements that must be met by the system. The quality requirements have been quantified in order to provide the SoftServe group and the client with a means to test the system for non-functional compliance. This allows the SoftServe group to set up tests in the non-functional testing document that can be used at the end of the development cycle to verify that the system complies with the architectural and quality requirements specified by this document. Further the document provides the architecture specification which covers architectural descriptions of various architectural factors that the project will need to follow when developing the system.  Architecture in this document's context refer to the technological basis and software development and organisational styles and strategies of the project. Thus this document serves as a contract between SoftServe and the client, Mrs Cathy Sandis of the DRIS of the University of Pretoria in terms of what technologies the project should incorporate as-well as the software development infrastructure that the system will be based on and also the quality the system should be of once the project is complete.
\vspace{0.2in}

\subsection{Documentation methodology}
\vspace{0.2in}
\begin{flushleft}
The documentation and software development methodology used by the project adhere to the guidelines set out by the scum agile methodology. Thus this document has undergone and will undergo various iterations that may extend or reduce the contents of the document.\\

This document was created using various insights gained throughout the COS 301 course and the requirement elicitation techniques and definitions as specified by Klaus Pohl’s book Requirements Engineering: Fundamentals, Principles, and Techniques [Dr.Phol, K., 2010].
The quality requirements and aspects of the architectural requirements, were elicited from the following sources:
\begin{itemize}
	\item Numerous interviews with the client.
	\item On-line research into UP Post doctoral applications.
	\item Correspondence with the UP IT department.
	\item Collecting and analysing various documents such as:
		\begin{itemize}
			\item The initial project request document
			\item Application forms
			\item Renewal forms
			\item CV templates
			\item Approval and recommendation forms
		\end{itemize}
\end{itemize}
\end{flushleft}	

\vspace{0.5in}

\subsection{Document conventions:}
\vspace{0.1in}
\begin{itemize}
\item Documentation formulation tool: LaTeX
\end{itemize}

\vspace{0.2in}

\subsection{References:}
\vspace{0.1in}
\begin{itemize}
\item Kang, A. August 9, 2002, \textit{Enterprise application integration using J2EE}, Available from: http://www.javaworld.com/article/2074488/enterprise-java/enterprise-application-integration-using-j2ee.html , [Accessed on: 17 May 2014]
\item Ali Babar, M., \textit{Architectural Patterns and Frameworks, Week 3, Lecture 3},Available from: https://blog.itu.dk/MSAR-E2013/files/2013/09/wk3\_lect3\_patternsframeworktactics.pdf, [Accessed on: 21 May 2014]
\item Dr.Phol, K., 2010, \textit{Requirements Engineering: Fundamentals, Principles, and Techniques}, Springer, Heidelberg.
\item Kayal, D., 2008, \textit{Pro Java EE Spring Patterns: Best Practices and Design Strategies Implementing Java EE patterns with Spring Framework}, Apress, New York.
\item Jendrock E, Cervera-Navarro R, Evans I, Haase K, Markito W, \textit{The Java EE 7 Tutorial}, Available from: http://docs.oracle.com/javaee/7/tutorial/doc/home.htm , [Accessed on: 20 May 2014]
\item Oracle. April 2014, \textit{The Java EE 7 Tutorial}, Available from: http://docs.oracle.com/javaee/7/tutorial/doc/overview.htm , [Accessed on: 20 May 2014]
\end{itemize}	

\vspace{0.5in}

\newpage
\section{Architecture requirements}

\subsection{Architectural Scope}
This section discusses the scope that the software architecture of the system needs to cover:
\begin{itemize}
\item A persistence infrastructure using a DBMS to facilitate the storage of the various domain objects (e.g CVs, DRIS information, and Applications). So to allow the system to store and centralise the various data that the system will be handling as outlined in the functional requirements and application document.
\item A multi-user session infrastructure to assist in realizing the security requirements of authenticating users and their actions.
\item A work-flow infrastructure for processing applications that can be modified and extended. 
\item A RESTful web service infrastructure, to provide a means for the system to be integrated in other systems and also to allow lightweight data communication between the users and the server and support for mobile platforms. 
\item A report generation infrastructure in order to support the data retrieval requirements of the client.
\item An email infrastructure for the system to use in order to provide notification support.
\item A action logging infrastructure to track user activity on the system, in order to provide audit-ability.
\item A gateway infrastructure to provide access control to services provided by the system.

\end{itemize}

\newpage
\subsection{Access channel requirements}
\vspace{0.2in}
The only two access channel requirements specified by the client are as follows:
\begin{itemize}

\item \textbf{All stakeholders:}
Users of the system will access it through the HTTP and HTTPS protocols either via a web browser client or some software, which makes use of the Restful API the system provides. Both will be locally installed on a user's desktop or mobile platform. Web browsers are expected to be HTML 5 compliment. This access channel will provide excellent availability and usability of the system due to the RESTFul nature of most internet enabled systems and the ease of using web browsers. The access channel will allow different stakeholders access to different sections of the system based on the roles assigned to their accounts. 

\item\textbf{System Administrator:}
The system administrator will require the same as the above but will in addition need to be able to conduct any external administration duties on the system which can not be done via the the above access channel. Such as updates, server maintenance, etc. Thus such operations will be done via the server by some terminal emulation protocol, such as Telnet or SSH, or by physical access. 

\end{itemize}
\vspace{0.2in}

\subsection{Quality requirements}
\vspace{0.2in}

\subsubsection{Security requirements}

\begin{flushleft}

The system will need to be fully secured since the system deals with confidential information such as person information, application statuses, financial data and meeting information. Also since the systems main goal is to provide stable and audible application and renewal process flow the system may not be vulnerable to data tampering or any tampering whatsoever. \\
\vspace{0.1in}

The system will have to provide different security roles to the registered users on the system. Any number of roles should be assignable to any user by a administrator with the correct role to allow for flexibility.
But in essence a stakeholder may only have access to their section of the application process. The system administrator should be able to view all the sections in the system and should be able to modify them except where they may not.

\end{flushleft}

\vspace{0.1in}

\subsubsection{Availability:}

\begin{flushleft}

The system's availability on designated networks will depend on the availability of the University of Pretoria's servers that host the system. If the University of Pretoria's servers hosting the system are active and provide access over a designated network then the system must be available over that designated network. The designated networks are defined as the internet and the campus network of the University of Pretoria.

\end{flushleft}

\vspace{0.1in}

\subsubsection{Testability:}

\begin{flushleft}

The system must be testable. This will be done using unit testing and following the test plan that will be laid out in the testing document of this project.\\

\vspace{0.1in}

Unit testing will test each unit in regard to:
\begin{itemize}

\item\textbf{Preconditions}
\item\textbf{Post conditions}

\end{itemize}

The project will also have two phases of testing:

\begin{itemize}

\item\textbf{Offline:} This is the initial phase of testing and debugging which will be done with pseudo data.
\item\textbf{Online:} This is the final phase of testing and debugging which will be done with active real time data.

\end{itemize}

\end{flushleft}

\vspace{0.1in}

\subsubsection{Scalability requirements}

\begin{flushleft}

The current aim is to create a scalable system that can support 500 to 1000 applicants per year with possible growth. This is in line with the figures given by the client and the growth in the research sector of the university.\\
\vspace{0.05in}

The system needs to be scalable in regard to the following factors:
\begin{itemize}


\item\textbf{Performance:} This is regarded as the speed and responsiveness of the system.
The system needs to be handle report queries in less than 10 seconds. It should be able to handle any application section processing in less than 3 seconds.\\

\item\textbf{Storage:} This is regarded as the growth and shrinking of the data that is stored.
The system will need to be able to handle a database that is in the range of 1 GB to 15 GB that has the potential to grow even larger. The reason for this stems from the requirement that the system will support archival functionality and archived data will store the data for long periods of time.\\

\item\textbf{Concurrency:} This regarded as the amount of active users on the system at the same time.
The system will need to support at least 100+/- concurrent users efficient and effectively since the system requires multiple stakeholders to part take in the application process while there can be multiple applications occurring at the same time.\\

\end{itemize}
\end{flushleft}
\vspace{0.1in}


\subsubsection{Auditability:}

\begin{flushleft}

The system needs to provide an audit trail of all critical actions that occur in the system. Critical actions are considered: user account management operations, login action, logout action and any operation by a user that leads to a change in application data of a particular prospective fellow.\\

\vspace{0.1in}

The Audit trail will be in the form of a read-only table stored in the database. It can only be viewed by a user with the correct security role. The system is the only entity that can modify the audit trail where this modification can only be the addition of entries.

\end{flushleft}
\vspace{0.1in}	

\subsubsection{Usability requirements}

\begin{flushleft}

The primary language of the system will be South African English. Any other language support is not considered part of the requirements but the system will be designed to allow for such development in the future.\\

\vspace{0.1in}

The system's UI will only consider 2 types of user categories with regard to usability:

\begin{itemize}

\item\textbf{Trained user:}

This type of user will have to have training in understanding how the system functions and how to use it. Their computer skills will be assumed to be in the range of basic to intermediate. Thus the user interface can allow for certain complexities but these complexities must be kept at a minimal. This user will be regarded as a system administrator. The stakeholders who fall under this category is the DRIS staff members overseeing the application process.

\item\textbf{Normal user:}

This type of user will have no training. Their computer skills will be assumed to be none or minimal. Therefore the UI that they will have access to will be simplistic and will be as user friendly as possible. The stakeholders that fall under this category will be Prospective fellows, Grant holders, HODs, Deans and Post-doctoral committee members.

\end{itemize}

\end{flushleft}

\vspace{0.2in}

\subsubsection{Robustness:}

\begin{flushleft}

The system needs to be robust, with the following areas of focus. It will have the ability to detect if the incorrect input has been entered into the system, the user will be notified immediately and not allowed to continue. If there are any defects with any other components with in the system and a task cannot be completed the user will also be notified immediately with the appropriate message sent to them. Users will also re-authenticate using a session object. Emails will be used for this authentication.
\end{flushleft}

\vspace{0.1in}

\vspace{0.2in}

\subsubsection{Flexibility:}

\begin{flushleft}

The system is designed in such a way that adding new features can be done without extensive code restructuring. The process of adding new features will require the programmer to create a new module for the new functionality.

\end{flushleft}

\vspace{0.1in}

\vspace{0.2in}

\subsubsection{Reusability:}

\begin{flushleft}

The System's components will be used again where applicable. The Login credentials is an example of areas where the system will be reused.

\end{flushleft}

\vspace{0.1in}

\vspace{0.2in}

\subsubsection{Maintability:}

\begin{flushleft}
The system administrator will have the responsibility of adding updates that a programmer may wish to add to the system. 

\end{flushleft}

\vspace{0.1in}

\vspace{0.1in}

\newpage
\subsection{Integration requirements}
\vspace{0.2in}
The client's primary integration requirement is to completely integrate with UP's PeopleSoft management system. With the regards the vision and scope document it should be noted that the knowledge the SoftServe group has of the Peoplesoft system is very limited due to UP's IT department's prerequisite for the project. The system has to integrate with the PeopleSoft as follows:

\begin{itemize}
	\item Must be able to create exportable data packages containing the information of prospective and current research fellows that can easily be loaded into the PeopleSoft system or used by other parties. In terms of quality requirements:
	\begin{enumerate}
		\item \textbf{Performance:} The system should be able to handle the export and importing of application information at a reasonable rate of about at least 10 full application data packages in a second. The system should be able to import and generate on-demand user accounts at a rate of 50 per second. For exporting it should be able to do so at 60 accounts per second.
		\item \textbf{Scalability:} The system should be able to export or import as much data as specified by the user as long as it does not exceed the size of the database or the size of users secondary storage. 
		\item \textbf{Audibility:} Any export and import should be recorded by the system.
	\end{enumerate}  
	\item Must be able to query PeopleSoft system to retrieve the account information of internal individual or group Stakeholders to create a uniform login experience. As specified by the vision and scope document this will not be possible at the time of writing since no API can be given to the SoftServe group. 
	\item To standardise information the following is to be done:
	\begin{enumerate}  
		\item Must assign Prospective fellows with a id number in such a way that is integrable with UP Emplid. SoftServe suggests: "f" + 8 digit code.
		\item Stakeholders that have a Peoplesoft account must use their UP emplid id as a user name. The password may vary.
		\item Must be able to provide access to the NRF researcher ratings list but not utilize it as the client has stipulated human error is possible in maintaining of those ratings.
	\end{enumerate}
\end{itemize}

SoftServe has also decided to add two further integration requirements in an attempt to improve future integration with other systems and to improve the application process. The two requirements are as follows:

\begin{itemize}
	\item \textbf{RESTful API/Web services} - The system will need to be able to provide a set of RESTful web services that is able to make use of all the back-end services provided by the system. This will allow future systems to easily integrate with other systems who wish to make use of the system. The interchange formats for the data should be limited to XML and JSON as these are the most universal and at the time of writing the de facto data formats. This will also help solve any future integration with the PeopleSoft system.
	\item \textbf{Google Scholar} - The system will need to be able to integrate with the Google Scholar services. The service needs to allow the system to be able to create full fledged queries for google scholar and be able to retrieve their results. This will be used to allow applicants to quickly populate their CV references by having the system quick search google scholar. There is unfortunately at the time of writing no such API. Therefore the SoftServe group will need to create our own API. This proposed by employing a HTML scrapping technique. 
\end{itemize} 

\vspace{0.2in}

\subsection{Architecture constraints}
\vspace{0.2in}

The following architecture constraints have been selected by SoftServe group as being suitable for the system. Please note due to the client's inexperience with software development and the technologies that exist she has allowed the SoftServe group to use their expertises, to decide on the appropriate set of architectural constraints that can meet the above requirements.
\begin{enumerate}
\item Database technology: MySQL
\item Programming paradigm: Object-Oriented
\item Programming language: Java
\item Development platform: Java-EE 7 
\item System architecture: Java-EE
\item Architectural frameworks: JSF, PAM, Java Persistence API, JAX-RS
\item Architectural pattern: Layered (Multi-tiered)
\item Development technologies: EJB, JSP, Expression Language, JDBC, JTA, Java EE Connector Architecture, JAF, JavaMail, JNDI, JAAS, Concurrency Utilities for Java EE, JAXB, PrimeFaces
\item IDE: Netbeans 8.0
\item Build environment: Apache Maven 3
\item Web server software: GlassFish 4.0 server
\item Web interface protocol: HTTPS
\item Client web browser: Microsoft Internet explorer 9+, Google Chrome 30+, Mozilla FireFox 20+, Opera, Safari
\item Client device operating systems: Windows, iOS, Android, Linux. 
\end{enumerate}
\vspace{0.5in}


\section{Architectural Patterns and Styles} % skipped till further notice.

% Mention of how Java EE actually doesn't require use of GoF design patterns >> http://www.theserverside.com/tip/With-Java-EE-7-your-Design-Patterns-are-dead-And-your-EAR-is-ugly-too <<

The system will employ a multi-tier/layered architectural pattern to form what is known as the Java Enterprise Edition system architecture. This will allow the client(s) to be decoupled from the server. This pattern is widely used and provides various benefits. Some benefits include modularity, encapsulation, re-usability of components, decoupling, system maintainability, pluggable-ness of layers and complexity reduction [Ali Babar, M.] [Oracle, 2014][Kayal, D., 2008]\\
\\
The Java EE system architecture is designed to support highly scalable, distributed, transactional and portable applications that use the speed, security, and reliability of a Java EE server to provide powerful enterprise applications. [Oracle, 2014] For this reason SoftServe believes this system architecture is well suited for the development of the Post-Doctoral application management system as it will satisfy the requirements in terms of quality and functional as stipulated by the client. The following diagram provides the system architecture that will be employed by the system:\\

\begin{figure}[H]
\centering
\includegraphics[scale=0.6]{../Images_Docs/Diagrams/Architecture/Java EE system architecture.jpg}
\caption{Java EE system architecture with respect to the Post-Doctoral application management system}
\end{figure}

\subsection{Client Tier}
This tier runs on the client system and encapsulates the various components that a client system may use to access the Java EE server-side tiers. These components include dynamic web pages, Java applications and Java applets. In order to make the Post-doctoral application management system accessible to any stakeholder over the internet and provide a uniform user experience the system will make use of the JSF frame work to provide web pages via a web browsers and RESTful web-services to provide access to the system via other client side applications. [Oracle, 2014]  

\subsection{Web Tier}
This Tier runs on the Java EE server and hosts the Web container and RESTful web services. It provides the management and web page generation support for the web pages that the system has to provide to the client Tier through the use of of the Java ServerFaces Facelets and backing beans. It also facilitates the communication between the business tier and client tier and thus handles the data validation and transformation required by such communication.[Oracle, 2014] 
\subsection{Business Tier}
This Tier also runs on the Java EE server and hosts the Enterprise Java Bean container. It provides the business logic section for the Java EE application in the form of Enterprise Java Beans which are simply classes of POJOs that represent various persistence entities of the persistence layer, system messages, sessions, services. It also hosts any auxiliary services that the application wishes to implement in order to help the EJBs. This tier communicates with EIS tier in order provide access to the database and various other lower level infrastructures that the Java EE application requires. [Oracle, 2014]

\subsection{Enterprise Information Tier} 
The EIS tier provides mainly the support for database systems that are used by the Java-EE application. This tier can run on the Java-EE server as a virtual server or on a physically different database server. Due to the project budget and technical constraints the former will used by the system. [Oracle, 2014] 
 
%%%%%%%%%%%%%%%%%%%%%%%%%%%%%%%%%%%%%%%%%%%%%%%%%%%%%%%%%%%%%%%%%%%%%%%%%%%%%%%%%%%%%%%%%%%%%%%%%%%%%%%%%%%%%
\section{Architectural Tactics and Strategies} % skipped till further notice.
This section describes the architectural techniques which will be used in order for the system to satisfy the quality requirements. 

\subsection{Concurrency}
The Java Platform offers support for concurrent programming, which is the basis for implementing many of the services offered by Java EE containers via concurrency utilities. This is realized through the use of thread pooling, threads, queuing and scheduling strategies.(Cervera-Navarro, Evans, Jendrock, Haase and Markito 2014).\\

\subsection{Exception handling}
To improve the robustness and reliability of the system the system employees various usages of try catch control structures and exception classes. This helps prevent system failure and improves availability. Also user friendly messages will be displayed to users using a messing framework to notify users about system failure or exceptions.

\subsection{Logging}
To improve security, audibility, reliability and maintainability the system logs exceptions and various other information such as queries and system status on a server log for later review. Thus it can be used to pinpoint bugs or attacks.

\subsection{Testing framework}
In order to satisfy reliability a testing framework will be used by the system in which unit and integration tests are designed and automatically ran at build time. To complement this a mock frame work will also be used to mock out dependencies for unit tests.

\subsection{Checkpoints and roll-backs}
To prevent the data integrity from being compromised when storing various items to the database in single operation the system will make use of two-phase commits and a entity manager that will only commit in a single transaction once all the data has been created/removed/modified in the temporary entity graph. Thus if an error occur all changes are rolled back to the previous checkpoint.

\subsection{Maintaining backups}
To improve reliability of the system the system will allow for full backups of the entire working database to be created.

\subsection{Interceptors}
To improve security and audibility interceptors will be employed by the system to detect the execution of methods that are demarcated by appropriate annotations. These interceptors will process the inputs to see if the user can use the service or if it needs to be logged by system in the working database. 

\subsection{Minimize access channels}
Another step taken to improve the security of the system was to minimize the access channels to only that of the web browser interface and RESTful API. Thus allowing the focus of security implementations on those provides.


\subsection{Design Patterns used}
To help improve maintainability and design of the system various design patterns will be employed.

\subsubsection{Builder Pattern}
The Builder Design Pattern allows for the construction of complex structures in small steps. An example of how it is used is in the process of making an application, the different elements of an application are separated. The process of creating objects speeds up as well. This construction of objects in small steps decreasing coupling and makes the code more testable. The design helps improve the flexibility as well as the maintainability of the code.  

\subsubsection{Factory Design Pattern}
The Factory Design Pattern provides centralised to create our Database entities and entries in an orderly fashion. The pattern allows the developers to define an interface for creating an object, but let the classes that implement the interface decide which class to instantiate. By doing this the code is now flexible and creation of new objects is much simpler. It also helps redundancy 

\subsubsection{State Design Pattern}
The State Design Pattern is best used in situations where the actions take place in a pre-defined order as with this project. The State pattern changes the state of the objects depending on how far in the pipeline. It provides a simple and clean way to change the state of an object during run time. The design pattern improves the usability of the system as users will know exactly when some change has taken place. 

\subsubsection{Dependency Injection}
This is used to implement, inversion control. The client is only allowed to use a service rather than creating their own services. The design pattern allows a client to remove all knowledge of a concrete implementation that it needs to use. This helps isolate the client from the impact of design changes and defects. It promotes re-usability, testability and maintainability 

\subsubsection{Database Access Objects}
The Database Access Object is used by all the application calls to provide specific data operations without exposing details of the database to external objects. It forms part of the Core J2EE Patterns. This design pattern acts an intermediary between the application and database, by moving back and forth between objects and database records. It also contains the effects of any changes to the persistence mechanism to a confined area and not the whole application. This pattern improves the audit-ability as well maintaining the integrity of the data contained in the system. 

The use of DAO will help reduce the existance of object-relational impedance mismatch present in the system. It will also contribute to the flexibilty of the system. In the case where the systems underlying persistance mechanism has to change only the DAO will have to be updated and all the places in the system where the DAO was used will then remain constant.

%%%%%%%%%%%%%%%%%%%%%%%%%%%%%%%%%%%%%%%%%%%%%%%%%%%%%%%%%%%%%%%%%%%%%%%%%%%%%%%%%%%%%%%%%%%%%%%%%%%%%%%%%%%%%
\section{Use of Reference Architecture and frameworks}
The core design philosophy of the Java-EE platform is to provide a Java-EE application developer with a set of test and well maintained and reusable APIs as well as frameworks that allow them to focus rather on implementing the actual business logic and UI than focusing on the underlying system's technical and management services such as authentication, session management, etc. Thus the Java EE platform provides a runtime environment for developing and running large-scale, multi-tiered, scalable, reliable, and secure network applications. As seen above it also provides a system architecture and various frameworks, discussed below, for implementing services for multi-tier applications that deliver the scalability, accessibility, and manageability needed by a system. Taking all the above into consideration this makes it ideal for the development of the project.

\subsection{JavaServer Faces (JSF)}
JSF is a web application GUI framework that is based on the JSP, EL and servlet technology that Java-EE provides. It allows the generation of various mark-up languages, such as HTML 4.0.1 and HTML 5, directly from objects and ORM model objects used by the Java-EE application. Thus it is ideal for system as the system needs to provide support for both HTML 5 and 4.0.1 web content.\\

It will help achieve the usability quality requirement as it will implement all aspects of the user interface.

\subsection{Java Persistence API}
Java EE is based on Java which is an object-oriented language. Whereas most modern day database management systems, DBMSs, provide relation databases. Thus to bridge this gap the Java Persistence API is used. It provides a Object Relational Mapping solution which allows the relation database to be viewed as a object-oriented database. Thus this critical for the system SoftServe wishes to develop as the system will make use of MySQL which is a relation database management system. The Java persistence API contains the following components:  
\begin{itemize}
\item Persistence API
\item The query language
\item ORM
\end{itemize}

\subsection{Java API for RESTful Web Services (JAX-RS)}
The JAX-RS API provides a way for the Java-EE application to provide web services or data transfer via the HTTP or HTTPS protocol using the Representational State Transfer, REST, architectural style. This accomplished by the user of various JAX-RS runtime annotations. This will allow the system to provide a set of lightweight web services to various clients across the internet. This will allow the system to easily be accesses by mobile and computer platforms alike and also be accessible over most companies firewalls as it will make use of the HTTPS protocol. Further this will allow for future expansion if client wishes it. To insure security a the POST command will be preferred above GET.[Cervera-Navarro, Evans, Jendrock, Haase and Markito, 2014]
\subsection{JUnit}
JUnit is a simple unit testing framework used to write repeatable tests. Test methods must be annotated by the @Test annotation. It is also possible to define a method to execute before (or after) each (or all) of the test methods with the @Before (or @After) and @BeforeClass (or @AfterClass) annotations. \\

It will be used to achieve the testbility of the system.
%\subsubsection{Java EE Connector Architecture}
%%%%%%%%%%%%%%%%%%%%%%%%%%%%%%%%%%%%%%%%%%%%%%%%%%%%%%%%%%%%%%%%%%%%%%%%%%%%%%%%%%%%%%%%%%%%%%%%%%%%%%%%%%%%%

\section{Access and Integration channels}
This section discusses the requirements for the access channels through which the system can be accessed by humans and other systems. Also it discusses the integration channels which need to be used by the system. 

\subsection{Access Channels}
To provide a system that is as accessible as possible the system provide its services through this essentially makes the system OS independent and accessible over firewalls since it will make use of the HTTPS protocol that is usually not blocked by firewalls. This also improves the usability the system as most computer user or mobile users are aware of how to use web browsers. The system will support the following versions of modern web browsers for both their computer and mobile counterparts:
\begin{enumerate}
\item Mozilla Firefox 20+
\item Google Chrome 30+
\item Microsoft Internet Explorer 9+
\item Apple Safari
\item Opera
\end{enumerate}

\subsection{Integration Channels}
As mentioned in the architecture requirements specification document the IT department of the University of Pretoria will only be willing to provide the SoftServe group with the knowledge of the Peoplesoft system in order to integrate it. Thus the SoftServe group will attempt to accommodate the PeopleSoft system as much as possible by conforming to same user name styling, export of data to formats that are used by the Peoplesoft system to import data. Also the since Peoplesoft is an Oracle enterprise system it would have been developed using Java-EE and the Java programming language. This is therefore a strong motivation for Java-EE to be used by the SoftServe group as a development platform for the system. So to allow integration at a more technical level. Though if the integration was to be done this would be done via the EIS teir using the Java EE Connector Architecture API and the JAX-WS API.  


%Upon successful complementation the system must be integrated able with the University of Pretoria's PeopleSoft system. Therefore it is sensible to implement the system in Java EE as it is one of the many components that form part of the multiple tiers that build up  PeopleSoft. \\

%The use of build tools such as Maven will provide a central piece of information with regards to the projects build, reporting and documentation. This central point of information will assist in the integration phase since it is implemented through a standardised build approach which can come into play at a later time through the use of EAI.\\

%The EAI will be achieved via  the logical integration architecture of Direct point-to-point integration (Kang, 2002). This means that the application management system will make direct JDBC calls to the universities databases tables (which needed to be setup to cater for our system at that point). The Integration method will be pushed-based data-level integration (or if all else fails UI-Level integration).
% http://www.javaworld.com/article/2074488/enterprise-java/enterprise-application-integration-using-j2ee.html

\section{Technologies}
This section discusses and elaborates on the technologies that the system will use and should also be seen as an extension of the architectural constraints, specified in the Architecture requirements specification document, in terms of elaboration.

\subsection{Integrated Development Environment}
The system should be buildable independent of an IDE but it will be developed on Netbeans 8.0 to allow for uniformity amongst the development team, with regards to coding style, and provide easy integration with the tools that will be used such as Javadoc, to generate  API documentation in HTML format.

\subsection{Build Tools}
\begin{itemize}
\item \textbf{Apache Maven} - This build tool was chosen due to the flexibility and power of it in terms of configurations, dependencies management, its ability to automate tests and the fact the project can easily be ported over to other IDEs such as the Netbeans IDE or even no IDE. 
\end{itemize}

\subsection{Server Operating System}

The system will be deployed on a single OS but will have clients that will use a variety of OSs. This is one of the main reasons the why the application's UI must be web browser based so to allow OS independent support. Thus the only prerequisite of the client's OS is that it needs to support a HTML 4 or 5 web browser and be internet accessible.
\begin{itemize}
\item \textbf{Server OS:}
	\begin{enumerate}
		\item Linux: Kubuntu 13.10
	\end{enumerate}
\item \textbf{Client OSs}
	\begin{enumerate}
		\item Windows: All
		\item Linux: All
		\item iOS: All
		\item Android: All
	\end{enumerate}
\end{itemize}

\subsection{Development technologies}
\subsubsection{Java Servlet Technology}
\textbf{Description}\\
A Java servlet is used to extend the Java-EE application server to support HTTP or HTTPS requests and responses. It allows the server to provide RESTful based web services to connecting clients. Thus it acts as a middle man between any HTTP or HTTPS client and the business tier. This will not be used directly but will be used the JSF framework.\\\\
\textbf{Reasons for use}\\
The JSF framework uses servlets to render JSF pages to HTML and provide them to clients connecting via HTTP or HTTPS. Secondly this will allow the solution to provide lightweight RESTful based web services that will help improve accessibility and availability of the solution. Further it will allow the solution to satisfy the access channel requirements.

\subsubsection{JavaServer Pages (JSP)}
\textbf{Description}\\
JSP is a technology that is used by the Java-EE platform to provide a native language approach to creating web pages. It uses HTML or XML to specify static content on a web page and Expression Language (EL) to provide dynamic content. This will not be used directly but will be used by the JSF framework.\\\\
\textbf{Reasons for use}\\
JSF pages is the a specialised version of JSP pages thus it is used in the JSF framework. Secondly it will help with the maintainability of the code as pages based on JSP technology are easily understandable and readable due to the simplicity of the HTML and XML mark-up languages.    
 
\subsubsection{Expression Language (EL)}
\textbf{Description}\\
It is a language used by JSP and JSF pages to write servlet based code snippets which allow the usage of the data available to the servlet to do calculations, call functions, get or set data. The language closely resembles the Java Language syntax.\\\\
\textbf{Reasons for use}\\
It is used by the JSP technology and JSF framework for dynamic content specification and communication with the backing servlet.

\subsubsection{JavaMail API}
\textbf{Description}\\
The JavaMail API provides a robust and well tested email communication infrastructure for any Java based applications.\\\\
\textbf{Reasons for use}\\
This technology will be employed by the system in order to provide the functional email notification infrastructure requirement needed by the solution.

\subsubsection{JavaBeans Activation Framework (JAF)}
\textbf{Description}\\
The JavaBeans Activation Framework allows the Java-EE application to determine the data type of some section of data and thus allow the application to provide access to it by encapsulating it and determining the operations that the application may perform on it.\\\\
\textbf{Reasons for use}\\
This technology is used by the JavaMail API thus it will be used by the system. 

\subsubsection{Java Database Connectivity API (JDBC)}
\textbf{Description}\\
The Java Database Connectivity API provides the a Java-EE application with the means to access data from various datasources including databases, spreadsheet, etc via the Java programming language.\\\\
\textbf{Reasons for use}\\
The system will use this to communicate with the databases located in the EIS tier of the system in order to retrieve and store data in the databases. 

\subsubsection{Java Transaction API (JTA)}
\textbf{Description}\\
The Java Transaction API provides the Java-EE Application with the means to handle and demarcate data transactions to the database. The API allows the the manual or automatic demarcation of database transactions and ensures that the database and ORM entities are synchronised after commits.\\\\ 
\textbf{Reasons for use}\\
This will be used in the system to improve the centralisation and accuracy of the data accessed by users. Further it will be used to allow access to multiple databases located in the EIS tier and the controlling of such resources.

\subsubsection{Enterprise JavaBeans (EJB)}
\textbf{Description}\\
The Enterprise JavaBeans is a component used by all Java-EE applications to encapsulate the various business logic of the application into reusable modules. Thus it contains the attributes and methods associated with the business logic module and hence can be treated as an stand alone unit that can be reused. The two primary EJBs are Session beans that represent a clients session and the data associated with it and a Message-driven bean that can a allow the component to receive messages asynchronously via event listeners. The system will make use of Stateless session EJBs in order to capture the back end services of the solution.\\\\
\textbf{Reasons for use}\\
The use of EJBs will ensure the modularity of the system and the re-usability of its components. Also within the Java-EE framework it is considered the core of the business tier and thus will be required.
  

\subsubsection{Java Naming and Directory Interface (JNDI)}
\textbf{Description}\\
The Java Naming and Directory Interface provides Java-EE applications with the ability to search for data across LDAP, DNS, etc, directory services. Above this it also allows the application to search for objects that exist within the application and provides access to them so that they can be used by the application. It is also used by Java-EE applications to locate object instances of EJBs and managed beans for usage by the application.\\\\
\textbf{Reasons for use}\\
This technology is a core service required by Java-EE applications to function and thus is needed by the solution.

\subsubsection{Java Architecture for XML Binding (JAXB)}
\textbf{Description}\\
JAXB is used to parse any XML data into a set of XML usable Java object instances that represent the content of the XML data. Likewise it is used to convert such objects to XML formatted data. This is accomplished by using an XML schema to define the structure or annotated classes.\\\\
\textbf{Reasons for use}\\
This will be used by the system to handle any XML data extraction or creation from or for storage.

\subsubsection{Dynamic Jasper Reports}
\textbf{Description}\\
The Dynamic Jasper reporting library provides a flexible, well-tested and maintained reporting framework to create reports based on data that is located in various data sources.\\\\
\textbf{Reasons for use}\\
This will be used to provide a reporting infrastructure as needed by the system on a functional requirement level.   

\subsubsection{Jsoup}
\textbf{Description}\\
Java library for working with real-world HTML. It provides a very convenient API for extracting and manipulating data, using the best of DOM, CSS, and jquery-like methods.\\\\
\textbf{Reasons for use}\\
Used to provide a pipe into the Google Scholar results.  

\subsubsection{JSF Managed Beans}
\textbf{Description}\\
This is a standard POJO (Plain Old Java Object), which is used to provide encapsulated services to  front-end JSF pages. JSF access them via Expression Lanaguage. These beans can be used in conjunction with EJBs and CDI to communicate with the database or call back-end services of hosted by the server.\\\\
\textbf{Reasons for use}\\
This technology is part of the JSF framework and is therefore required. It further also allows improved modularity and re-usability of components.

\subsubsection{Primefaces}
\textbf{Description}\\
This is component library which is used to expand and provide an improved range of easy to use components for JSF pages. It incorporates Javascript, jQuery and AJAX in order to provide the various components.\\\\
\textbf{Reasons for use}\\
It is an easy to use, versatile library that allows the solution provide a sophisticated and clean user interface. Further is highly compatible with mobile platform HTTP web browsers thus allowing the solution to provide better accessibility without extensive modification.    

\subsubsection{Mockito}
\textbf{Description}\\
This is a mock framework that allows the mocking of dependencies during unit testing phases.\\\\
\textbf{Reasons for use}\\
It is easy to use and provides a great number of effective features that will help enhance unit testing during the development of the solution. Thus it helps improve the testability of the solution.

\subsubsection{MySQL DBMS}
\textbf{Description}\\
This is a well-proven open-source relational database management system that provides extensive set of features for maintaining the data and database.\\\\
\textbf{Reasons for use}\\
It a is well-proven DBMS. Further since the data captured by the solution will fit well in a relation approach the DBMS will be efficient enough to meet the performance and scalability requirements of the solution.



\newpage


\newpage
\section{Glossary:}
\vspace{0.2in}

\begin{itemize}

\item \textbf{API} - Application Programming Interface
\item \textbf{Application} -Both renewal applications or new fellowship applications are seen as applications by this project.
\item \textbf{CV} - Curriculum Vita
\item \textbf{EAI} - Enterprise Application Integration
\item \textbf{NRF} - National Research Foundation
\item \textbf{Spreadsheet} - A special type of computer document that is used to represent data in rows and columns.
\item \textbf{GlassFish} - GlassFish is a web server software package that is very flexible and compatible with Java EE applications. 
\item \textbf{HTML} - Hyper Text Mark-up Language
\item \textbf{HTTPS} - Hyper Text Transfer Protocol Secure is a higher level network oriented communication rule set that is highly secure and is used by all web browsers. 
\item \textbf{Java EE} - Java Enterprise Edition
\item \textbf{MySQL} - Is a relational persistence database package that provides all the necessary management tools to run and manage a database server.
\item \textbf{Object-Oriented} - A programming language style that encapsulates everything as an object instance of a particular class of attributes and methods.
\item \textbf{JDBC} - Java Database Connection
\item \textbf{MVC} - Model View Controller
\item \textbf{UI} - User Interface
\item \textbf{UP} - University of Pretoria
\item \textbf{Application} - Both a renewal and new fellowship are seen as applications.

\end{itemize}	


\vspace{0.5in}


\end{document}