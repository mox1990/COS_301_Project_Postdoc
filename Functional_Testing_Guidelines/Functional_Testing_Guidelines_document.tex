\documentclass[12pt]{article}

\usepackage[english]{babel}
\usepackage[utf8x]{inputenc}
\usepackage{pdfpages}
\usepackage{lastpage} % Required to determine the last page for the footer
\usepackage{extramarks} % Required for headers and footers
\usepackage{graphicx} % Required to insert images
\usepackage{listings} % Required for insertion of code
\usepackage{courier} % Required for the courier font
\usepackage{color}
\usepackage{grffile}
\usepackage{float}

\usepackage[a4paper, total={6in, 8in}]{geometry}

% Margins
\topmargin=-0.45in
\evensidemargin=0in
\oddsidemargin=0in
\textwidth=6.5in
\textheight=9.0in
\headsep=0.25in
\fboxsep=0mm%padding thickness
\fboxrule=2pt%border thickness

\linespread{1.1} % Line spacing

\newcommand{\Title}{Functional testing document} % Assignment title
\newcommand{\Class}{COS\ 301 Final year project} % Course/class
\newcommand{\pd}{Post-Doctoral}
\newcommand{\ssr}{Soft\color{green}{Serve }\color{black}}
\newcommand{\version}{0.9}
\newcommand{\iteration}{4}
\newcommand{\client}{Ms. Cathy Sandis (UP Research Office)}
\newcommand{\project}{Post-Doctoral Application Management System}

\begin{document}


\vspace{4em}

\begin{center}%

\begin{figure}[ht!]
\centering
\includegraphics{../Images_Docs/logo.png}
\end{figure}
\LARGE \bf \Class \\[0.25em]
\LARGE \bf \project \\[1em]
\LARGE \bf \Title \\[0.25em]
\large \bf \today\\
\bf Version \version\\
\bf Iteration \iteration\\[0.5em]
\Large \bf Prepared for \client\\
\Large \bf by
\Large {\bf \ssr Group }\\[0.5em]
\LARGE {\bf Group members}\\[0.25em]
\large
Kgothatso Phatedi Alfred Ngako (12236731) \\[0.5em]
Tokologo “Carlo” Machaba (12078027) \\[0.5em]
Mathys Ellis (12019837) \\[8em]

\end{center}%

%\newpage
%{\LARGE \bf Change log}\\[2em]

\begin{center}
\begin{tabular}{|l|p{1.4cm}|p{8cm}|p{2.8cm}|}
\hline
\multicolumn{4}{|c|}{\bf Change log} \\
\hline
 Date & Version & Description &  Person \\
\hline
10/02/2014 & v 0.0 & Original SRS document created & Mathys Ellis \\
\hline
02/03/2014 & v 0.1 & Added to glossary & Mathys Ellis \\
\hline
05/03/2014 & v 0.3 & Added Introduction, Vision, Background & Carlo Machaba \\
\hline
06/03/2014 & v 0.4 & Added open issues. Modified some sections & Alfred Ngako \\
\hline
06/03/2014 & v 0.5 & Added methodology, scope and limitations & Mathys Ellis \\
\hline
08/03/2014 & v 0.6 & Added some wrapping to the change log which is now a table & Alfred Ngako \\
\hline
16/03/2014 & v 0.8 & Did some restructuring and document formatting & Mathys Ellis \\
\hline
17/03/2014 & v 0.8 & Also added to the glossary & Mathys Ellis \\
\hline
12/05/2014 & v 0.9 & Created new vision and scope document. Transferred necessary content from old SRS document. Performed editing and restructuring of document. Added exclusions & Mathys Ellis \\
\hline

%\end{tabbing}
\end{tabular}
\end{center}
\newpage
\tableofcontents

\listoffigures
\newpage
\section{Document description:}

\subsection{Document purpose:}
\vspace{0.2in}
This vision and scope document serves the purpose of providing a detailed overview of the project's scope and its vision as well the goals that SoftServe's Post-Doctoral application management system wishes to satisfy. Further it defines the abstract interaction of stakeholders with the proposed software system. Thus this document serves as a contract between SoftServe and the client, Mrs Cathy Sandis of the DRIS of the University of Pretoria in terms of project scope.

\vspace{0.2in}

\subsection{Documentation methodology}
\vspace{0.2in}
\begin{flushleft}
The documentation and software development methodology used by the project adhere to the guidelines set out by the agile method. Thus this document has undergone and will undergo various iterations that may extend or reduce the contents of the document.\\

This document was created using the requirement elicitation techniques and requirement definitions as specified by Klaus Pohl’s book Requirements Engineering: Fundamentals, Principles, and Techniques [Dr.Phol, K., 2010].
The requirements, vision and scope were elicited from the following sources:
\begin{itemize}
	\item Numerous interviews with the client.
	\item On-line research into UP Post doctoral applications.
	\item Correspondence with the UP IT department.
	\item Collecting and analysing various documents such as:
		\begin{itemize}
			\item The initial project request document
			\item Application forms
			\item Renewal forms
			\item CV templates
			\item Approval and recommendation forms
		\end{itemize}
\end{itemize}
\end{flushleft}	

\vspace{0.5in}

\subsection{Document conventions:}
\vspace{0.1in}
\begin{itemize}
\item Documentation formulation tool: LaTeX
\item ERD Crow-Foot notation
\item UML 2.0
\end{itemize}

\vspace{0.2in}
\subsection{References:}
\vspace{0.1in}
\begin{itemize}
\item Dr.Phol, K., 2010, \textit{Requirements Engineering: Fundamentals, Principles, and Techniques}, Springer, Heidelberg.
\item Yvonne Rogers, Helen Sharp,  Jenny Preece,\textit{  Interaction Design: Beyond Human–Computer Interaction, 3rd ed.}, John Wiley, Sons Ltd., 2011 ISBN 0-470-66576-9
\end{itemize}	

\vspace{0.5in}


\newpage
\section{Testing Methodology}
We obtained the tests through the means of graphs which followed all the possible paths that can be followed by using the service, the pre and post conditions stated in the Functional Requirements Document. As a result of this process mentioned above, we can assume that the test cover all the functions as stated by the Functional Requirement Documentation. The section below lists all the unit tests as well as the a description of what the function does. 


\section{Unit Testing}
 
\subsection{ApplicationProgressViewer}

\begin{center}
\begin{tabular}{|l|p{12cm}|}
\hline
 Name: & testGetApplicationProgressWithUserAsOwnerAndApplicationOpen  \\
\hline
Level: & Unit \\
\hline
TestCode: & v APV \\
\hline
Module:& Application Progress Viewer \\
\hline
Function & APV \\
\hline
Description: & To test EJB's function to get the progress of an open application which belongs to the current user. \\
\hline
%\end{tabbing}
\end{tabular}
\end{center}

\begin{center}
\begin{tabular}{|l|p{12cm}|}
\hline
 Name: & testGetApplicationProgressWithUserAsOwnerAndApplicationOpen  \\
\hline
Level: & Unit \\
\hline
TestCode: & v APV \\
\hline
Module:& Application Progress Viewer \\
\hline
Function & APV \\
\hline
Description: & To test EJB's function to get the progress of an open application which belongs to the current user. \\
\hline
%\end{tabbing}
\end{tabular}
\end{center}

\subsection{Application Services Util}

\begin{center}
\begin{tabular}{|l|p{12cm}|}
\hline
 Name: & testLoadPendingApplications  \\
\hline
Level: & Unit \\
\hline
TestCode: & v APSU \\
\hline
Module:& Application Progress Viewer \\
\hline
Function & APSU \\
\hline
Description: & To test the EJB's function to get the all pending applications that belong to current user. \\
\hline
%\end{tabbing}
\end{tabular}
\end{center}

\begin{center}
\begin{tabular}{|l|p{12cm}|}

\hline
 Name: & testGetTotalNumberOfPendingApplicationsWithStatusAndReferre  \\
\hline
Level: & Unit \\
\hline
TestCode: & v APSU \\
\hline
Module:& Application Progress Viewer \\
\hline
Function & APSU \\
\hline
Description: & To test the EJB's function to get the number all pending applications that have specified person as referre. \\
\hline
%\end{tabbing}
\end{tabular}
\end{center}

\begin{center}
\begin{tabular}{|l|p{12cm}|}

\hline
 Name: & testGetTotalNumberOfPendingApplicationsWithStatusAndGrantholder  \\
\hline
Level: & Unit \\
\hline
TestCode: & v APSU \\
\hline
Module:& Application Progress Viewer \\
\hline
Function & APSU \\
\hline
Description: & To test the EJB's function to get the number of all pending applications that have the specified person as grantholder. \\
\hline
%\end{tabbing}
\end{tabular}
\end{center}

\begin{center}
\begin{tabular}{|l|p{12cm}|}
\hline

 Name: & testGetTotalNumberOfPendingApplicationsWithStatusAndDepartment  \\
\hline
Level: & Unit \\
\hline
TestCode: & v APSU \\
\hline
Module:& Application Progress Viewer \\
\hline
Function & APSU \\
\hline
Description: & To test the EJB's function to get the number of all pending applications that fall under specified department. \\
\hline
%\end{tabbing}
\end{tabular}
\end{center}

\begin{center}
\begin{tabular}{|l|p{12cm}|}
\hline

 Name: & testGetTotalNumberOfPendingApplicationsWithStatusAndFaculty  \\
\hline
Level: & Unit \\
\hline
TestCode: & v APSU \\
\hline
Module:& Application Progress Viewer \\
\hline
Function & APSU \\
\hline
Description: & To test the EJB's function to get the number of all pending applications that fall under the specied faculty. \\
\hline
%\end{tabbing}
\end{tabular}
\end{center}

\begin{center}
\begin{tabular}{|l|p{12cm}|}
\hline

 Name: & testGetTotalNumberOfPendingApplications  \\
\hline
Level: & Unit \\
\hline
TestCode: & v APSU \\
\hline
Module:& Application Progress Viewer \\
\hline
Function & APSU \\
\hline
Description: & To test the EJB's function to get the number of all pending applications that belong to specified user. \\
\hline
%\end{tabbing}
\end{tabular}
\end{center}

\begin{center}
\begin{tabular}{|l|p{12cm}|}
\hline

 Name: & testDeclineAppliction  \\
\hline
Level: & Unit \\
\hline
TestCode: & v APSU \\
\hline
Module:& Application Progress Viewer \\
\hline
Function & APSU \\
\hline
Description: & To test the EJB's function to allow authorized user to decline an application. \\
\hline
%\end{tabbing}
\end{tabular}
\end{center}

\begin{center}
\begin{tabular}{|l|p{12cm}|}
\hline

 Name: & testDeclineApplictionAlreadyDeclined  \\
\hline
Level: & Unit \\
\hline
TestCode: & v APSU \\
\hline
Module:& Application Progress Viewer \\
\hline
Function & APSU \\
\hline
Description: & To test the EJB's function to behave when an application has already been declined. \\
\hline
%\end{tabbing}
\end{tabular}
\end{center}


\subsection{Audit Trail}

\begin{center}
\begin{tabular}{|l|p{12cm}|}
\hline

 Name: & testLogAction  \\
\hline
Level: & Unit \\
\hline
TestCode: & v APV \\
\hline
Module:& Application Progress Viewer \\
\hline
Function & APV \\
\hline
Description: & To test EJB's function to Log an action to the system. \\
\hline
%\end{tabbing}
\end{tabular}
\end{center}

\subsection{CV Management}

\begin{center}
\begin{tabular}{|l|p{12cm}|}
\hline

 Name: & testGetApplicationProgressWithUserAsOwnerAndApplicationOpen  \\
\hline
Level: & Unit \\
\hline
TestCode: & v CV \\
\hline
Module:& Application Progress Viewer \\
\hline
Function & CV \\
\hline
Description: & To test EJB's function to get the progress of an open application which belongs to the current user. \\
\hline
%\end{tabbing}
\end{tabular}
\end{center}


\begin{center}
\begin{tabular}{|l|p{12cm}|}
\hline

 Name: & testCreateCVButHasCV  \\
\hline
Level: & Unit \\
\hline
TestCode: & v APV \\
\hline
Module:& Application Progress Viewer \\
\hline
Function & CV \\
\hline
Description: & Test EJB's function to behave when a CV trying to create a CV that already exists. \\
\hline
%\end{tabbing}
\end{tabular}
\end{center}

\begin{center}
\begin{tabular}{|l|p{12cm}|}
\hline

 Name: & testUpdateCV  \\
\hline
Level: & Unit \\
\hline
TestCode: & v CV \\
\hline
Module:& CV Management \\
\hline
Function & CV \\
\hline
Description: & Test EJB's function to update an existing CV. \\
\hline
%\end{tabbing}
\end{tabular}
\end{center}


\subsection{DRIS Approval}

\begin{center}
\begin{tabular}{|l|p{12cm}|}
\hline
 Name: & testLoadPendingEndorsedApplications  \\
\hline
Level: & Unit \\
\hline
TestCode: & v DA \\
\hline
Module:& DRIS Approval \\
\hline
Function & DA \\
\hline
Description: & To test EJB's function to get the all pending applications that fall under a specified range. \\
\hline
%\end{tabbing}
\end{tabular}
\end{center}

\begin{center}
\begin{tabular}{|l|p{12cm}|}
\hline

 Name: & testCountTotalPendingEndorsedApplications  \\
\hline
Level: & Unit \\
\hline
TestCode: & v DA \\
\hline
Module:& DRIS Approval \\
\hline
Function & DA \\
\hline
Description: & To test EJB's function to get the number of all pending applications. \\
\hline
%\end{tabbing}
\end{tabular}
\end{center}

\begin{center}
\begin{tabular}{|l|p{12cm}|}
\hline

 Name: & testLoadPendingEligibleApplications  \\
\hline
Level: & Unit \\
\hline
TestCode: & v DRIS \\
\hline
Module:& Application Progress Viewer \\
\hline
Function & DRIS \\
\hline
Description: & To test EJB's function to get the all pending applications that fall under a specified range. \\
\hline
%\end{tabbing}
\end{tabular}
\end{center}

\subsection{Deans Endorsement}

\begin{center}
\begin{tabular}{|l|p{12cm}|}
\hline

 Name: & testLoadPendingEligibleApplications  \\
\hline
Level: & Unit \\
\hline
TestCode: & v DE \\
\hline
Module:& DeansEndorsement \\
\hline
Function & DE \\
\hline
Description: & To test EJB's function to get the all applications which fall under a specied range. \\
\hline
%\end{tabbing}
\end{tabular}
\end{center}

\begin{center}
\begin{tabular}{|l|p{12cm}|}
\hline

 Name: & testCountTotalPendingApplications  \\
\hline
Level: & Unit \\
\hline
TestCode: & v DE \\
\hline
Module:& DeansEndorsement \\
\hline
Function & DE \\
\hline
Description: & To test EJB's function to get the nubmer of all applications which a pending approval. \\
\hline
%\end{tabbing}
\end{tabular}
\end{center}

\begin{center}
\begin{tabular}{|l|p{12cm}|}
\hline

 Name: & testDenyApplication  \\
\hline
Level: & Unit \\
\hline
TestCode: & v DE \\
\hline
Module:& DeansEndorsement \\
\hline
Function & DE \\
\hline
Description: & To test EJB's function to decline a specific application. \\
\hline
%\end{tabbing}
\end{tabular}
\end{center}

\subsection{Fast Forward And Rewind Service}

\begin{center}
\begin{tabular}{|l|p{12cm}|}
\hline

 Name: & testForwardApplication  \\
\hline
Level: & Unit \\
\hline
TestCode: & v FFR \\
\hline
Module:& Fast Forward And Rewind Service \\
\hline
Function & FFR \\
\hline
Description: & Test EJB's function to Forward an application\\
\hline
%\end{tabbing}
\end{tabular}
\end{center}

\begin{center}
\begin{tabular}{|l|p{12cm}|}
\hline

 Name: & testRewindApplication  \\
\hline
Level: & Unit \\
\hline
TestCode: & v FFR \\
\hline
Module:& Fast Forward And Rewind Service \\
\hline
Function & FFR \\
\hline
Description: & Test EJB's function to Rewind an application\\
\hline
%\end{tabbing}
\end{tabular}
\end{center}

\begin{center}
\begin{tabular}{|l|p{12cm}|}
\hline

 Name: & testLoadMovableApplication  \\
\hline
Level: & Unit \\
\hline
TestCode: & v FFR \\
\hline
Module:& Fast Forward And Rewind Service \\
\hline
Function & FFR \\
\hline
Description: & Test EJB's function to show the applications which can be moved\\
\hline
%\end{tabbing}
\end{tabular}
\end{center}

\subsection{Grant Holder Finalisation}

\begin{center}
\begin{tabular}{|l|p{12cm}|}
\hline

 Name: & testCreateGrantHolderCV  \\
\hline
Level: & Unit \\
\hline
TestCode: & v GHF \\
\hline
Module:& Grant Holder Finalisation \\
\hline
Function & GHF \\
\hline
Description: & Test EJB's function to create a new CV for the grantholder. \\
\hline
%\end{tabbing}
\end{tabular}
\end{center}

\begin{center}
\begin{tabular}{|l|p{12cm}|}
\hline

 Name: & testCreateGrantHolderCVNotValid  \\
\hline
Level: & Unit \\
\hline
TestCode: & v GHF \\
\hline
Module:& Grant Holder Finalisation \\
\hline
Function & GHF \\
\hline
Description: & Test EJB's function behavoiur to create a new CV for the grantholder when a CV already exist. \\
\hline
%\end{tabbing}
\end{tabular}
\end{center}

\begin{center}
\begin{tabular}{|l|p{12cm}|}
\hline

 Name: & testFinaliseApplication  \\
\hline
Level: & Unit \\
\hline
TestCode: & v GHF \\
\hline
Module:& Grant Holder Finalisation \\
\hline
Function & GHF \\
\hline
Description: & Test EJB's function to finalise an application. \\
\hline
%\end{tabbing}
\end{tabular}
\end{center}

\begin{center}
\begin{tabular}{|l|p{12cm}|}
\hline

 Name: & testFinaliseApplicationWithNotifications  \\
\hline
Level: & Unit \\
\hline
TestCode: & v GHF \\
\hline
Module:& Grant Holder Finalisation \\
\hline
Function & GHF \\
\hline
Description: & Test EJB's function to finalise an application and send notifications in the same procedure. \\
\hline
%\end{tabbing}
\end{tabular}
\end{center}

\subsection{HOD Recommendation}

\begin{center}
\begin{tabular}{|l|p{12cm}|}
\hline

 Name: & testDenyAppliction  \\
\hline
Level: & Unit \\
\hline
TestCode: & v HOD \\
\hline
Module:& HODRecommendation \\
\hline
Function & HOD \\
\hline
Description: & To test EJB's function to allow a HOD to deny a grant holder. \\
\hline
%\end{tabbing}
\end{tabular}
\end{center}


\begin{center}
\begin{tabular}{|l|p{12cm}|}
\hline

 Name: & testApproveApplicationWithoutDeansToEndorse  \\
\hline
Level: & Unit \\
\hline
TestCode: & v HOD \\
\hline
Module:& HODRecommendation \\
\hline
Function & HOD \\
\hline
Description: & To test EJB's function to allow a HOD approve an application. \\
\hline
%\end{tabbing}
\end{tabular}
\end{center}


\begin{center}
\begin{tabular}{|l|p{12cm}|}
\hline

 Name: & testAmmendAppliction  \\
\hline
Level: & Unit \\
\hline
TestCode: & v HOD \\
\hline
Module:& HODRecommendation \\
\hline
Function & HOD \\
\hline
Description: & To test EJB's function to allow a HOD to ammend an application. \\
\hline
%\end{tabbing}
\end{tabular}
\end{center}

\subsection{Location Management}
\begin{center}
\begin{tabular}{|l|p{12cm}|}
\hline

 Name: & testCreateInstitution\\
 \hline
Level: & Unit \\
\hline
TestCode: & v LM1 \\
\hline
Module:& Location Management \\
\hline
Function & LM1 \\
\hline
Description: & To test out the ability to create an institution  \\
\hline
\end{tabular}
\end{center}

\begin{center}
\begin{tabular}{|l|p{12cm}|}
\hline

 Name: & testCreateFaculty\\ 
 \hline
Level: & Unit \\
\hline
TestCode: & v LM2 \\
\hline
Module:& Location Management \\
\hline
Function & LM2 \\
\hline
Description: & To test out the ability to create an institution  \\
\hline
\end{tabular}
\end{center}

\begin{center}
\begin{tabular}{|l|p{12cm}|}
\hline

 Name: & testCreateDepartment\\
 \hline
Level: & Unit \\
\hline
TestCode: & v LM3 \\
\hline
Module:& Location Management \\
\hline
Function & LM3 \\
\hline
Description: & To test out the ability to create a department  \\
\hline
\end{tabular}
\end{center}

\begin{center}
\begin{tabular}{|l|p{12cm}|}
\hline

 Name: & testUpdateInstitution\\
 \hline
Level: & Unit \\
\hline
TestCode: & v LM4 \\
\hline
Module:& Location Management \\
\hline
Function & LM4 \\
\hline
Description: & To test out the ability to update an institution  \\
\hline
\end{tabular}
\end{center}

\begin{center}
\begin{tabular}{|l|p{12cm}|}
\hline

 Name: & testUpdateDepartment \\
 \hline
Level: & Unit \\
\hline
TestCode: & v LM5 \\
\hline
Module:& Location Management \\
\hline
Function & LM5 \\
\hline
Description: & To test out the ability to update a department  \\
\hline
\end{tabular}
\end{center}

\begin{center}
\begin{tabular}{|l|p{12cm}|}
\hline

 Name: & testGetInstitution \\
 \hline
Level: & Unit \\
\hline
TestCode: & v LM6 \\
\hline
Module:& Location Management \\
\hline
Function & LM6 \\
\hline
Description: & To test out the ability to get an institution  \\
\hline
\end{tabular}
\end{center}

\begin{center}
\begin{tabular}{|l|p{12cm}|}
\hline

 Name: & testGetAllFacultiesAtAnInstitution \\
 \hline
Level: & Unit \\
\hline
TestCode: & v LM7 \\
\hline
Module:& Location Management \\
\hline
Function & LM7 \\
\hline
Description: & To test out the ability to get all the faculties at an institution  \\
\hline
\end{tabular}
\end{center}

\begin{center}
\begin{tabular}{|l|p{12cm}|}
\hline

 Name: & testGetAllDepartmentForAnInstitution \\
 \hline
Level: & Unit \\
\hline
TestCode: & v LM8 \\
\hline
Module:& Location Management \\
\hline
Function & LM8 \\
\hline
Description: & To test out the ability to get all departments at an institution  \\
\hline
\end{tabular}
\end{center}

\begin{center}
\begin{tabular}{|l|p{12cm}|}
\hline

 Name: & testGetInstitution \\
 \hline
Level: & Unit \\
\hline
TestCode: & v LM9 \\
\hline
Module:& Location Management \\
\hline
Function & LM9 \\
\hline
Description: & To test out the ability to get an institution  \\
\hline
\end{tabular}
\end{center}

\begin{center}
\begin{tabular}{|l|p{12cm}|}
\hline

 Name: & testGetFaculty \\
 \hline
Level: & Unit \\
\hline
TestCode: & v LM10 \\
\hline
Module:& Location Management \\
\hline
Function & LM10 \\
\hline
Description: & To test out the ability to get a faculty \\
\hline
\end{tabular}
\end{center}

\begin{center}
\begin{tabular}{|l|p{12cm}|}
\hline

 Name: & testGetDepartment \\
 \hline
Level: & Unit \\
\hline
TestCode: & v LM11 \\
\hline
Module:& Location Management \\
\hline
Function & LM11 \\
\hline
Description: & To test out the ability to get a department \\
\hline
\end{tabular}
\end{center}

\subsection{Meeting Management}
\begin{center}
\begin{tabular}{|l|p{12cm}|}
\hline

 Name: & testAddMeetingComments  \\
\hline
Level: & Unit \\
\hline
TestCode: & v MM1 \\
\hline
Module:& Meeting Management \\
\hline
Function & MM1 \\
\hline
Description: & To test out the ability to add meeting comments  \\
\hline

%\end{tabbing}
\end{tabular}
\end{center}

\begin{center}
\begin{tabular}{|l|p{12cm}|}
\hline

 Name: & testCreateMeeting  \\
\hline
Level: & Unit \\
\hline
TestCode: & v MM2 \\
\hline
Module:& Meeting Management \\
\hline
Function & MM2 \\
\hline
Description: & To test out the ability to create meetings \\
\hline

%\end{tabbing}
\end{tabular}
\end{center}

\begin{center}
\begin{tabular}{|l|p{12cm}|}
\hline

 Name: & testEndMeeting  \\
\hline
Level: & Unit \\
\hline
TestCode: & v MM3 \\
\hline
Module:& Meeting Management \\
\hline
Function & MM3 \\
\hline
Description: & To test out the ability to end meetings   \\
\hline

%\end{tabbing}
\end{tabular}
\end{center}

\begin{center}
\begin{tabular}{|l|p{12cm}|}
\hline

 Name: & testStartMeeting  \\
\hline
Level: & Unit \\
\hline
TestCode: & v MM4 \\
\hline
Module:& Meeting Management \\
\hline
Function & MM4 \\
\hline
Description: & To test out the ability to start meetings  \\
\hline

%\end{tabbing}
\end{tabular}
\end{center}

\begin{center}
\begin{tabular}{|l|p{12cm}|}
\hline

 Name: & testUpdateMeetingWithoutAttendance  \\
\hline
Level: & Unit \\
\hline
TestCode: & v MM5 \\
\hline
Module:& Meeting Management \\
\hline
Function & MM5 \\
\hline
Description: & To test out the ability to update meetings without any attendance  \\
\hline
%\end{tabbing}
\end{tabular}
\end{center}

\subsection{New Application}
\begin{center}
\begin{tabular}{|l|p{12cm}|}
\hline

 Name: & testCanFellowOpenApplication  \\
\hline
Level: & Unit \\
\hline
TestCode: & v NA1 \\
\hline
Module:& New Application\\
\hline
Function & NA1 \\
\hline
Description: & To test out see if the fellow can open a new application  \\
\hline

%\end{tabbing}
\end{tabular}
\end{center}

\begin{center}
\begin{tabular}{|l|p{12cm}|}
\hline

 Name: & testCreateNewApplication  \\
\hline
Level: & Unit \\
\hline
TestCode: & v NA2 \\
\hline
Module:& New Application\\
\hline
Function & NA2 \\
\hline
Description: & To test out see if the fellow can  new application  \\
\hline

%\end{tabbing}
\end{tabular}
\end{center}

\begin{center}
\begin{tabular}{|l|p{12cm}|}
\hline
 Name: & testCreateNewApplicationNull  \\
\hline
Level: & Unit \\
\hline
TestCode: & v NA3 \\
\hline
Module:& New Application\\
\hline
Function & NA3 \\
\hline
Description: & To test out see if the fellow can  new application with a null value and if the exception is thrown  \\
\hline

%\end{tabbing}
\end{tabular}
\end{center}

\begin{center}
\begin{tabular}{|l|p{12cm}|}
\hline

 Name: & testCreateProspectiveFellowCV  \\
\hline
Level: & Unit \\
\hline
TestCode: & v NA4 \\
\hline
Module:& New Application\\
\hline
Function & NA2 \\
\hline
Description: & To test out see if the fellow can  create a CV  \\
\hline

%\end{tabbing}
\end{tabular}
\end{center}

\begin{center}
\begin{tabular}{|l|p{12cm}|}
\hline

 Name: & testLinkToGrantHolder  \\
\hline
Level: & Unit \\
\hline
TestCode: & v NA5 \\
\hline
Module:& New Application\\
\hline
Function & NA5 \\
\hline
Description: & To test out see if the grant holder has been added to the application  \\
\hline

%\end{tabbing}
\end{tabular}
\end{center}

\begin{center}
\begin{tabular}{|l|p{12cm}|}
\hline

 Name: & testLinkToGrantHolderNull  \\
\hline
Level: & Unit \\
\hline
TestCode: & v NA6 \\
\hline
Module:& New Application\\
\hline
Function & NA6 \\
\hline
Description: & To test out see if the grant holder has been added to the application with a null grant holder.  \\
\hline

%\end{tabbing}
\end{tabular}
\end{center}

\begin{center}
\begin{tabular}{|l|p{12cm}|}
\hline

 Name: & testLinkToReferee  \\
\hline
Level: & Unit \\
\hline
TestCode: & v NA7 \\
\hline
Module:& New Application\\
\hline
Function & NA7 \\
\hline
Description: & To test out see if the referee has been added to the application  \\
\hline

%\end{tabbing}
\end{tabular}
\end{center}

\begin{center}
\begin{tabular}{|l|p{12cm}|}
\hline

 Name: & testLinkToRefereeNull  \\
\hline
Level: & Unit \\
\hline
TestCode: & v NA8 \\
\hline
Module:& New Application\\
\hline
Function & NA8 \\
\hline
Description: & To test out see if the referee has been added to the application with null referee  \\
\hline

%\end{tabbing}
\end{tabular}
\end{center}

\begin{center}
\begin{tabular}{|l|p{12cm}|}
\hline

 Name: & testSubmitApplicationNull  \\
\hline
Level: & Unit \\
\hline
TestCode: & v NA9 \\
\hline
Module:& New Application\\
\hline
Function & NA9 \\
\hline
Description: & To test out see if the application has been submitted with null value  \\
\hline

%\end{tabbing}
\end{tabular}
\end{center}


\begin{center}
\begin{tabular}{|l|p{12cm}|}
\hline

 Name: & testSubmitApplication \\
\hline
Level: & Unit \\
\hline
TestCode: & v NA10 \\
\hline
Module:& New Application\\
\hline
Function & NA10 \\
\hline
Description: & To test out see if the application has been submitted   \\
\hline

%\end{tabbing}
\end{tabular}
\end{center}

\subsection{Neural Networks}

\begin{center}
\begin{tabular}{|l|p{12cm}|}
\hline

 Name: & testLoadAllNeuralNetworks \\
 \hline
Level: & Unit \\
\hline
TestCode: & v NN1 \\
\hline
Module:& Neural Networks \\
\hline
Function & NN1 \\
\hline
Description: & To test out the ability to load all the neural networks  \\
\hline
\end{tabular}
\end{center}

\begin{center}
\begin{tabular}{|l|p{12cm}|}
\hline

 Name: & testGetDefaultlNeuralNetwork \\
 \hline
Level: & Unit \\
\hline
TestCode: & v NN2 \\
\hline
Module:& Neural Networks \\
\hline
Function & NN2 \\
\hline
Description: & To test out the ability to get the default neural network  \\
\hline
\end{tabular}
\end{center}
\begin{center}
\begin{tabular}{|l|p{12cm}|}
\hline

 Name: & testCreateNeuralNetworks \\
 \hline
Level: & Unit \\
\hline
TestCode: & v NN3 \\
\hline
Module:& Neural Networks \\
\hline
Function & NN3 \\
\hline
Description: & To test out the ability to create a neural network  \\
\hline
\end{tabular}
\end{center}

\begin{center}
\begin{tabular}{|l|p{12cm}|}
\hline

 Name: & testMakeNeuralNetworkDefaultNetwork \\
 \hline
Level: & Unit \\
\hline
TestCode: & v NN4 \\
\hline
Module:& Neural Networks \\
\hline
Function & NN4 \\
\hline
Description: & To test out the ability to make the neural network the default network  \\
\hline
\end{tabular}
\end{center}

\begin{center}
\begin{tabular}{|l|p{12cm}|}
\hline

 Name: & testUpdateNeuralNetwork \\
 \hline
Level: & Unit \\
\hline
TestCode: & v NN5 \\
\hline
Module:& Neural Networks \\
\hline
Function & NN5 \\
\hline
Description: & To test out the ability to update the neural network  \\
\hline
\end{tabular}
\end{center}

\begin{center}
\begin{tabular}{|l|p{12cm}|}
\hline

 Name: & testUpdateNeuralNetworkSynapses \\
 \hline
Level: & Unit \\
\hline
TestCode: & v NN6 \\
\hline
Module:& Neural Networks \\
\hline
Function & NN6 \\
\hline
Description: & To test out the ability to update the neural network synapses  \\
\hline
\end{tabular}
\end{center}

\begin{center}
\begin{tabular}{|l|p{12cm}|}
\hline

 Name: & testRemoveNeuralNetwork \\
 \hline
Level: & Unit \\
\hline
TestCode: & v NN7 \\
\hline
Module:& Neural Networks \\
\hline
Function & NN7 \\
\hline
Description: & To test out the ability to remove a neural network   \\
\hline
\end{tabular}
\end{center}


\begin{center}
\begin{tabular}{|l|p{12cm}|}
\hline

 Name: & testCorrectNeuralNetwork \\
 \hline
Level: & Unit \\
\hline
TestCode: & v NN8 \\
\hline
Module:& Neural Networks \\
\hline
Function & NN8 \\
\hline
Description: & To test out the ability to correct a neural network   \\
\hline
\end{tabular}
\end{center}


\begin{center}
\begin{tabular}{|l|p{12cm}|}
\hline

 Name: & testTrainNeuralNetwork \\
 \hline
Level: & Unit \\
\hline
TestCode: & v NN9 \\
\hline
Module:& Neural Networks \\
\hline
Function & NN9 \\
\hline
Description: & To test out the ability to train a neural network   \\
\hline
\end{tabular}
\end{center}

\subsection{Notification}
\begin{center}
\begin{tabular}{|l|p{12cm}|}
\hline

 Name: & testFindAll  \\
\hline
Level: & Unit \\
\hline
TestCode: & v NT1 \\
\hline
Module:& Notification\\
\hline
Function & NT1 \\
\hline
Description: & To test out see if all the notification can be found \\
\hline

%\end{tabbing}
\end{tabular}
\end{center}

\begin{center}
\begin{tabular}{|l|p{12cm}|}
\hline

 Name: & testFindByTimeStamp  \\
\hline
Level: & Unit \\
\hline
TestCode: & v NT2 \\
\hline
Module:& Notification\\
\hline
Function & NT2 \\
\hline
Description: & To test out see if all the notification can be found by timestamp \\
\hline

%\end{tabbing}
\end{tabular}
\end{center}

\begin{center}
\begin{tabular}{|l|p{12cm}|}
\hline

 Name: & testFindByDate  \\
\hline
Level: & Unit \\
\hline
TestCode: & v NT3 \\
\hline
Module:& Notification\\
\hline
Function & NT3 \\
\hline
Description: & To test out see if all the notification can be found by date \\
\hline

%\end{tabbing}
\end{tabular}
\end{center}

\begin{center}
\begin{tabular}{|l|p{12cm}|}
\hline

 Name: & testFindByID  \\
\hline
Level: & Unit \\
\hline
TestCode: & v NT4 \\
\hline
Module:& Notification\\
\hline
Function & NT4 \\
\hline
Description: & To test out see if all the notification can be found by ID \\
\hline

%\end{tabbing}
\end{tabular}
\end{center}

\begin{center}
\begin{tabular}{|l|p{12cm}|}
\hline

 Name: & testFindByRange  \\
\hline
Level: & Unit \\
\hline
TestCode: & v NT5 \\
\hline
Module:& Notification\\
\hline
Function & NT5 \\
\hline
Description: & To test out see if all the notification can be found within a range \\
\hline

%\end{tabbing}
\end{tabular}
\end{center}

\begin{center}
\begin{tabular}{|l|p{12cm}|}
\hline

 Name: & testFindBySubject  \\
\hline
Level: & Unit \\
\hline
TestCode: & v NT6 \\
\hline
Module:& Notification\\
\hline
Function & NT6 \\
\hline
Description: & To test out see if all the notification can be found by subject \\
\hline

%\end{tabbing}
\end{tabular}
\end{center}

\begin{center}
\begin{tabular}{|l|p{12cm}|}
\hline

 Name: & testSendNotificationByEmail \\
\hline
Level: & Unit \\
\hline
TestCode: & v NT7 \\
\hline
Module:& Notification\\
\hline
Function & NT7 \\
\hline
Description: & To test out send notifications by email \\
\hline

%\end{tabbing}
\end{tabular}
\end{center}

\begin{center}
\begin{tabular}{|l|p{12cm}|}
\hline

 Name: & testSendBatchNotification \\
\hline
Level: & Unit \\
\hline
TestCode: & v NT8 \\
\hline
Module:& Notification\\
\hline
Function & NT8 \\
\hline
Description: & To test out send batch notifications by email \\
\hline

%\end{tabbing}
\end{tabular}
\end{center}

\begin{center}
\begin{tabular}{|l|p{12cm}|}
\hline

 Name: & testSendNotificationWithoutEmail \\
\hline
Level: & Unit \\
\hline
TestCode: & v NT9 \\
\hline
Module:& Notification\\
\hline
Function & NT9 \\
\hline
Description: & To test out send notifications without emails. \\
\hline

%\end{tabbing}
\end{tabular}
\end{center}
\subsection{Progress Reports}
\begin{center}
\begin{tabular}{|l|p{12cm}|}
\hline

 Name: & testCreateProgressReport \\
\hline
Level: & Unit \\
\hline
TestCode: & v PR1 \\
\hline
Module:& Progress Report\\
\hline
Function & PR1 \\
\hline
Description: & To test out if a porgress report has been created \\
\hline

%\end{tabbing}
\end{tabular}
\end{center}

\begin{center}
\begin{tabular}{|l|p{12cm}|}
\hline

 Name: & testSubmitProgressReport \\
\hline
Level: & Unit \\
\hline
TestCode: & v PR2 \\
\hline
Module:& Progress Report\\
\hline
Function & PR2 \\
\hline
Description: & To test out if a porgress report has been submitted \\
\hline

%\end{tabbing}
\end{tabular}
\end{center}

\subsection{Referee Report}
\begin{center}
\begin{tabular}{|l|p{12cm}|}
\hline

 Name: & testCreateRefereeReport \\
\hline
Level: & Unit \\
\hline
TestCode: & v RR1 \\
\hline
Module:& Report Report\\
\hline
Function & RR1 \\
\hline
Description: & To test out if a referee report has been created \\
\hline

%\end{tabbing}
\end{tabular}
\end{center}

\begin{center}
\begin{tabular}{|l|p{12cm}|}
\hline

 Name: & testSubmitRefereeReport \\
\hline
Level: & Unit \\
\hline
TestCode: & v PR1 \\
\hline
Module:& Referee Report\\
\hline
Function & PR1 \\
\hline
Description: & To test out if a referee report has been submitted \\
\hline

%\end{tabbing}
\end{tabular}
\end{center}

\subsection{User Account Management}
\begin{center}
\begin{tabular}{|l|p{12cm}|}
\hline

 Name: & testCreateUser \\
\hline
Level: & Unit \\
\hline
TestCode: & v UA1 \\
\hline
Module:& User Account Management\\
\hline
Function & UA1 \\
\hline
Description: & To test out if a new user has been created \\
\hline

%\end{tabbing}
\end{tabular}
\end{center}


\begin{center}
\begin{tabular}{|l|p{12cm}|}
\hline

 Name: & testCreateUserFalse \\
\hline
Level: & Unit \\
\hline
TestCode: & v UA2 \\
\hline
Module:& User Account Management\\
\hline
Function & UA1 \\
\hline
Description: & To test out if a new user has been created without systemID known\\
\hline

%\end{tabbing}
\end{tabular}
\end{center}

\begin{center}
\begin{tabular}{|l|p{12cm}|}
\hline

 Name: & testGenerateOnDemandAccount \\
\hline
Level: & Unit \\
\hline
TestCode: & v UA3 \\
\hline
Module:& User Account Management\\
\hline
Function & UA3 \\
\hline
Description: & To test out if an on demand account has been created \\
\hline

%\end{tabbing}
\end{tabular}
\end{center}

\begin{center}
\begin{tabular}{|l|p{12cm}|}
\hline

 Name: & testGetRemoveUserTrue \\
\hline
Level: & Unit \\
\hline
TestCode: & v UA4 \\
\hline
Module:& User Account Management\\
\hline
Function & UA4 \\
\hline
Description: & To test out if a user has been removed \\
\hline

%\end{tabbing}
\end{tabular}
\end{center}

\begin{center}
\begin{tabular}{|l|p{12cm}|}
\hline

 Name: & testUpdateUser \\
\hline
Level: & Unit \\
\hline
TestCode: & v UA5 \\
\hline
Module:& User Account Management\\
\hline
Function & UA5 \\
\hline
Description: & To test out if a user has been updated \\
\hline

%\end{tabbing}
\end{tabular}
\end{center}

\begin{center}
\begin{tabular}{|l|p{12cm}|}
\hline

 Name: & testViewAllAccounts \\
\hline
Level: & Unit \\
\hline
TestCode: & v UA7 \\
\hline
Module:& User Account Management\\
\hline
Function & UA7 \\
\hline
Description: & To test out if the system admin can view all accounts \\
\hline

%\end{tabbing}
\end{tabular}
\end{center}

\begin{center}
\begin{tabular}{|l|p{12cm}|}
\hline

 Name: & testGetRemoveUserFail \\
\hline
Level: & Unit \\
\hline
TestCode: & v UA8 \\
\hline
Module:& User Account Management\\
\hline
Function & UA8 \\
\hline
Description: & To test out if a user has been removed because an exception as expected \\
\hline

%\end{tabbing}
\end{tabular}
\end{center}

\begin{center}
\begin{tabular}{|l|p{12cm}|}
\hline

 Name: & testGenerateOnDemandAccountTrue\\
\hline
Level: & Unit \\
\hline
TestCode: & v UA9 \\
\hline
Module:& User Account Management\\
\hline
Function & UA9 \\
\hline
Description: & To test out if an on demand for the specified user \\
\hline

%\end{tabbing}
\end{tabular}
\end{center}

\begin{center}
\begin{tabular}{|l|p{12cm}|}
\hline

 Name: & testAddresses\\
\hline
Level: & Unit \\
\hline
TestCode: & v UA10 \\
\hline
Module:& User Account Management\\
\hline
Function & UA10 \\
\hline
Description: & To test out if the address for the specified user exists \\
\hline

%\end{tabbing}
\end{tabular}
\end{center}

\subsection{User Gateway}
\begin{center}
\begin{tabular}{|l|p{12cm}|}
\hline

 Name: & testAuthenticateUser \\
\hline
Level: & Unit \\
\hline
TestCode: & v UG1 \\
\hline
Module:& User Gateway\\
\hline
Function & UG1 \\
\hline
Description: & To test out if a user has been authenticated \\
\hline

%\end{tabbing}
\end{tabular}
\end{center}

\begin{center}
\begin{tabular}{|l|p{12cm}|}
\hline

 Name: & testGetSessionFromHttpSession \\
\hline
Level: & Unit \\
\hline
TestCode: & v UG1 \\
\hline
Module:& User Gateway\\
\hline
Function & UG1 \\
\hline
Description: & To test out if it  \\
\hline

%\end{tabbing}
\end{tabular}
\end{center}

\begin{center}
\begin{tabular}{|l|p{12cm}|}
\hline

 Name: & testLogin\\
\hline
Level: & Unit \\
\hline
TestCode: & v UG2 \\
\hline
Module:& User Gateway\\
\hline
Function & UG2 \\
\hline
Description: & To test out if a user has been logged in \\
\hline

%\end{tabbing}
\end{tabular}
\end{center}


\begin{center}
\begin{tabular}{|l|p{12cm}|}
\hline

 Name: & testLogout\\
\hline
Level: & Unit \\
\hline
TestCode: & v UG3 \\
\hline
Module:& User Gateway\\
\hline
Function & UG3 \\
\hline
Description: & To test out if a user has been logged out \\
\hline

%\end{tabbing}
\end{tabular}
\end{center}


\begin{center}
\begin{tabular}{|l|p{12cm}|}
\hline

 Name: & testAuthenticateUserAsOwner \\
\hline
Level: & Unit \\
\hline
TestCode: & v UG4 \\
\hline
Module:& User Gateway\\
\hline
Function & UG4 \\
\hline
Description: & To test out if a user has been authenticated as the owner \\
\hline

%\end{tabbing}
\end{tabular}
\end{center}

\newpage
\section{Integretation Testing}

\subsection{User Accounts and Notification Service}
\begin{center}
\begin{tabular}{|l|p{12cm}|}
\hline
 Name: & testNewUserAndNotification  \\
\hline
Level: & Integration \\
\hline
TestCode: & INUN \\
\hline
Module:& User Account Management and Notification Service \\
\hline
Function & INUN \\
\hline
Description: & This test ensures the integration between the User Account Management Service and Notification Service. \\
\hline
%\end{tabbing}
\end{tabular}
\end{center}

\subsection{Referral Reports and Notification Service}
\begin{center}
\begin{tabular}{|l|p{12cm}|}
\hline
 Name: & testReferralAndNotification  \\
\hline
Level: & Integration \\
\hline
TestCode: & IRAN \\
\hline
Module:& Referee Reports and Notification Service \\
\hline
Function & ICNA \\
\hline
Description: & This test ensures the integration between the Referee Report Service and Notification Service. \\
\hline
%\end{tabbing}
\end{tabular}
\end{center}

\subsection{Meeting Management and Notification Service}
\begin{center}
\begin{tabular}{|l|p{12cm}|}
\hline
 Name: & testMeetingAndNotification  \\
\hline
Level: & Integration \\
\hline
TestCode: & ICNA \\
\hline
Module:& Meeting Management and Notification Service \\
\hline
Function & ICNA \\
\hline
Description: & This test ensures the integration between the Meeting Management Service and Notification Service. \\
\hline
%\end{tabbing}
\end{tabular}
\end{center}

\subsection{Creating A New Application}
\begin{center}
\begin{tabular}{|l|p{12cm}|}
\hline
 Name: & testWorkFlow  \\
\hline
Level: & Integration \\
\hline
TestCode: & ICNA \\
\hline
Module:& New Application, Grant Holder's Report, Referee Report, HOD Recommendation, Dean's Endorsement, DRIS Approval \\
\hline
Function & ICNA \\
\hline
Description: & To test out the ability of a new application to move through the work flow required for the whole system to function. \\
\hline
%\end{tabbing}
\end{tabular}
\end{center}

\subsection{Applying for Renewal}

\begin{center}
\begin{tabular}{|l|p{12cm}|}
\hline
 Name: & testWorkFlowWithRenewal  \\
\hline
Level: & Integration \\
\hline
TestCode: & IAR \\
\hline
Module:&  New Application, Grant Holder's Report, Referee Report, HOD Recommendation, Dean's Endorsement, DRIS Approval\\
\hline
Function & IAR \\
\hline
Description: & To test out the ability of a renewal application to move through the work flow required for the whole system to function. \\
\hline
%\end{tabbing}
\end{tabular}
\end{center}


\subsection{Applying for Fellowship with Fast Forwarding and Rewind Service}

\begin{center}
\begin{tabular}{|l|p{12cm}|}
\hline
 Name: & testWorkFlowWithNewApplicationWithFastForward  \\
\hline
Level: & Integration \\
\hline
TestCode: & IAR \\
\hline
Module:&  New Application, Grant Holder's Report, Referee Report, HOD Recommendation, Dean's Endorsement, DRIS Approval\\
\hline
Function & IAR \\
\hline
Description: & To test out the ability of a renewal application to move through the work flow, but while fast forwarding and rewinding the application. \\
\hline
%\end{tabbing}
\end{tabular}
\end{center}

\subsection{Google Scholar}
\begin{center}
\begin{tabular}{|l|p{12cm}|}
\hline
 Name: & testGoogleScholarAPI \\
\hline
Level: & Integration \\
\hline
TestCode: & IAR \\
\hline
Module:&  New Application, Grant Holder's Report, Referee Report, HOD Recommendation, Dean's Endorsement, DRIS Approval\\
\hline
Function & IAR \\
\hline
Description: & To test out the ability of tehe EJB's to use scrapping to retrieve academic work from Google Scholar. \\
\hline
%\end{tabbing}
\end{tabular}
\end{center}

\newpage
\section{Usability Tests}
For the Usability Tests Softserve will be using the DECIDE Framework:
\subsection{DECIDE Framework}
\subsubsection{Determine the goals of evaluation}

A Post-Doctoral fellow is a person who conducts research after they have completed their PhD, with the aim of deepening their knowledge in a specified. The University of Pretoria supports such research opportunities in order to the increase research output of the University. Post-Doctoral fellows who conduct their research at the University of Pretoria do so under the supervision of a staff member of the University and their research may be privately or internally funded. This is a growing field in Universities around South Africa. A lack in the software solutions for the application management of Post-Doctoral fellows has been identified by the SoftServe group.

The purpose of the evaluation is to check areas of usability that are successfully comprehended by the user, and at the same time also discover those areas that are not consistent or intuitive for the user, all in efforts to better the system. The specific goal of the evaluation is to see if the users can intuitively navigate through the application to perform common tasks by understanding their role in the system. The evaluation will also consider the time it took to perform the tasks and will help discover better solutions for parts of the evaluation where the user struggled to use the application. The evaluation is important to the success of the system, as it is imperative that users be able to understand the options available to them and be able to efficiently navigate the system for their benefit. 

\subsubsection{Explore the specific questions to be answered}

Are users intuitively able to navigate through the application?
Are users able to perform tasks faster after using the application again?
Is the process of creating an application intuitive enough?
How do the users feel about the forms filling process?
General opinions on User Interface Elements?

\subsubsection{Choose the evaluation paradigm and techniques to answer questions}

The primary evaluation technique that was used for the evaluation was Usability Testing. The reason for choosing this technique is because the evaluations needed to take place in a controlled setting where the evaluators are able keep track of various behaviours of the users after performing predetermined tasks on the application.
The data gathering techniques used was taking notes, video recording and monitoring key strokes. These techniques were used to produce both quantitative and qualitative data but primarily quantitative data that can easily be quantified and summarized into meaningful averages that can be further analysed at a later stage.

\subsubsection{Identify the practical issues that must be addressed}
\begin{enumerate}

\item	Users will have to include current Post Doctoral students, DRIS members, the client and other computer literate members.
\item Users will have to concurrently be able to complete their tasks on the application with other users in the laboratory.
\item The users may be disturbed by external noise from nearby construction, students passing by etc., but should not affect the outcomes of the evaluation because the application was designed with simplicity and does not require a lot of focus.
\item The user only has an hour to complete all their tasks otherwise their results will have to be nullified. 
\item	Different users will have different tasks and as a result there could be distractions in the lab
\item	The users may be disturbed by external noise from nearby construction, students passing by etc… This may affect the outcomes of the evaluation because although the application was designed with simplicity and it does require a lot of focus.
\item Making the application accessible over the UP network.

\end{enumerate}

\subsubsection{Decide how to deal with the ethical issues}

Ethical issues, which we may encounter, would be dealing with confidentiality of information we receive from the users. We will address this issue by presenting a consent form for the use to sign – that guards their rights and our special privilege to their information. Our duty after that is to make sure their personal information is not compromised or released. Another issue, which we may encounter, would be dealing with the actual testing. The participants will be told in advance that if any part of the testing is not comfortable, they may choose to quit the testing. 

\subsubsection{Evaluate, analyse, interpret and present the data} 

We will evaluate the data looking for patterns in the user’s behaviour for certain tasks and take all users into consideration that gave us consent and completed all the tasks in time. We will analyse the data by using both quantitative and qualitative frameworks and techniques to get meaningful data that can help us in improving the application’s user experience and expectations. We will interpret the data to a human understandable format that allows an easier understanding evaluation taking place. We will then present the data in order to help detect where to improve or tweak the application to improve a user’s experience of using the application. 

\newpage
\subsection{Questionnaire}
The following is the questionnaire users will be filling out.
Using a scale of 1 to 5, where 1 is Strongly Disagree and 5 is Strongly Agree

\begin{center}
\begin{tabular}{|l|p{2cm}|}
\hline
 Question: & Rating \\
\hline
	Overall, I am satisfied with how easy it is to use the system. &  \\ \hline
	It was simple to use the system. & \\ \hline
	I would effectively complete the tasks using this system. & \\ \hline
	I was able to complete the tasks quickly using this system. & \\ \hline
	I was able to efficiently complete the tasks using the system. & \\ \hline
	I feel comfortable using the system. & \\ \hline
	It was easy to learn to use the system. & \\ \hline
	I believe I could become productive quickly using the system. & \\ \hline
	The system gave error messages that clearly told me how to fix the problem. & \\ \hline
	Whenever I made a mistake using the system, I could recover easily and quickly. & \\ \hline
	It was easy to find the information I needed. & \\ \hline
	The information provided for the system was easy to understand. & \\ \hline
	The information was effective in helping me complete the tasks. & \\ \hline
	The organization of information on the system screens was clear. & \\ \hline
	The interface of the system pleasant. & \\ \hline
 I liked using the interface of the system. & \\ \hline
	The system has all the functions and capabilities I expect it to have. & \\ \hline
	Overall, I am satisfied with the system. & \\ \hline

\end{tabular}
\end{center}
\begin{itemize}
\item	Please list three things you liked most about this system software.

\item	Please list three things you liked least about this system software.

\end{itemize}

\subsection{Results}
\begin{figure}[H]
\centering	
\framebox{\includegraphics[scale=0.75]{../Images_Docs/UsabilityResults.png}}
\caption{Table displaying the results from the tests}
\end{figure}

In essence, the users found the application easy to use as they could each complete the tasks they were required to complete successfully and with ease. The users further indicated that they were generally pleased and happy with the application.

However users felt that the application could have been designed to provide helpful error messages and a better way to navigate through the system.

With the results from our usability test we can say that our product gave the users a satisfactory user experience. Our product was able to achieve its intended use and would make the processes involved in the Post Doctoral Management easier to manage. Also, the product was designed well enough as users indicated in the satisfactory questionnaire and we were able to meet the usability goals. So our product would be good enough to undergo a field test/study (with some minor editions to it as pointed out by our users) 

\newpage


\section{Glossary:}
\vspace{0.2in}

\begin{itemize}

\item \textbf{Application} -Both renewal applications or new fellowship applications are seen as applications by this project.
\item \textbf{UP} - University of Pretoria
 


\end{itemize}	

\end{document}